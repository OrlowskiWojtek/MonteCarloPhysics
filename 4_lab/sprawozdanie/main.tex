\documentclass[a4paper, 12pt]{article}

\input{"$HOME/Desktop/Studia/LaTeX/setup.tex"}

\title{\textsc{Monte Carlo: proste całkowanie z szacowaniem wariancji}\\ - sprawozdanie}
\author{Wojciech Orłowski}
\date{\today}

\begin{document}
    \maketitle

    \section{Wstęp}
    
    Celem ćwiczenia jest oszacowanie metodą Monte Carlo pola wspólnego dwóch przekrywających się pól.
    Problem ten może być trudny do rozwiązania, a rozważania Monte Carlo pozwolą w łatwy sposób uzyskać podobne wyniki dla większej ilości wymiarów, gdzie rozważania geometryczne stają się jeszcze cięższe.
    W celu rozwiązania tego problemu metodą Monte Carlo należy generować jednorodnie punkty w kole.
    Taki generator został napisany na poprzednich zajęciach, a jego procedura jest następująca:
    \begin{enumerate}
        \item losowanie pary punktów $(x,y)$ o rozkładzie normalnym $N(0,1)$ korzystając z rozkładu jednorodnego $U_1,U_2 = U(0,1)$,
        \[ x = \sqrt{-2\ln(U1)}\sin(2\pi U2), \]
        \[ y = \sqrt{-2\ln(U1)}\cos(2\pi U2), \] 
        \item normalizacje wylosowanych punktów do okręgu
        \[ x = \frac{x}{\sqrt{x^2 + y^2}}, \] 
        \[ y = \frac{y}{\sqrt{x^2 + y^2}}, \] 
        \item przeskalowanie do rozkładu normalnego na kole
        \[ q = \sqrt{(U(0,1))}, \]
        \[ x = x \cdot q, \]
        \[ y = y \cdot q, \]
        \item przesunięcie oraz przeskalowanie do koła o innym promieniu
        \[ x = x \cdot R_\alpha + x_\alpha, \]
        \[ y = y \cdot R_\alpha + y_\alpha, \]
        gdzie $R_\alpha, x_\alpha, y_\alpha$ oznaczają promień oraz współrzędne koła końcowego.
    \end{enumerate}

    \noaka Rozkład jednorodny w kole musi spełniać warunek normalizacji. 
    Dla rozkładu jednorodnego funkcja gęstości prawdopodobieństwa (fgp) jest stała.
    Zatem fgp jest stała na całym kole i wynosi
    \[ f_\alpha(x,y) = \frac{1}{\pi R_\alpha^2}.  \] 
    Całkę (czyli powierzchnie) części wspólnej definiujemy jako 
    \[S_{\alpha,\beta} = \pi R^2_\alpha \iint_{x,y \in K_\alpha} \theta_{\alpha,\beta}(x,y) \; \dd x \dd y,  \]
    gdzie $\theta$, zależna jest od pary punktów $(x,y)$, i definiowana jest jako
    \[\theta_{\alpha, \beta}(x,y) = \begin{cases}
        1, (x,y) \in K_\alpha \land (x,y) \in K_\beta \\
        0, (x,y) \notin K_\alpha \lor (x,y) \notin K_\beta  \\ 
    \end{cases} \] 
    Gdzie właściwie wiedząc, że $(x,y)$ są przynależne do $K_\alpha$, wystarczy sprawdzić przynależność do $K_\beta$. 
    W ten sposób możemy skonstruować metodę Monte Carlo - losując $(x,y)$ z $K_\alpha$, i sprawdzając czy punkt ten przynależy do $K_\beta$.
    Pole powierzchni można obliczyć licząc tą całkę metodą Monte Carlo:
    \begin{equation}
        \mu^{(1)} = \bar{S}_{\alpha,\beta} = \frac{\pi R_\alpha^2}{N} \sum_{i=1}^N \theta_{\alpha,\beta}(x_i,y_i).
        \label{mu_1}
    \end{equation}
    Drugi moment, który będzie potrzebny do obliczenia drugiego momentu centralnego (wariancji) obliczymy poprzez:
    \[ \mu^{(2)} = \frac{1}{N} \sum_{i = 1}^{N} \left( \pi R_\alpha^2  \theta_{\alpha,\beta}(x_i,y_i) \right)^2, \]
    korzystając z faktu, że funkcja $\theta$ przyjmuje tylko wartości 0 i 1, powyższe wyrażenie można przepisać jako
    \begin{equation}
        \mu^{(2)} =  \pi R_\alpha^2 \mu^{(1)}.
        \label{mu_2}
    \end{equation} 
    Odchylenie standardowe jako parametr dokładności wyników zostaje policzone jako (za jego pomocą możemy podać dokładność zgodnie z testem statystycznym):
    \begin{equation}
        \sigma_{\bar{S}_{\alpha,\beta}} = \sqrt{\frac{\mu^{(2)} - (\mu^{(1)})^2}{N}}
        \label{std_dev}
    \end{equation}
    Podczas wykonywanego ćwiczenia pole wspólne będzie liczone losując z koła $K_\alpha$, a także $K_\beta$, co może wpłynąć na dokładność wyników. 

    \section{Wyniki}

    Dla generowanych punktów w kołach zostały założone wartości: 
    \begin{itemize}
        \item $R_\alpha =  2$,
        \item $R_\beta = 2\sqrt{2}$,
        \item $\vb{r_\alpha} = [x_\alpha, 0]$,
        \item $\vb{r_\beta} = [0,0] $,
    \end{itemize}
    $x_\alpha$ jest parametrem i będzie zmieniany w zależności od przeprowadzanego doświadczenia numerycznego.

    \subsection*{Test generatora}

    Dla testu wygenerowano $N = 10000$ punktów w dwóch kołach. 
    Parametr $x_\alpha$ został ustawiony na $R_\alpha + R_\beta$.
    Wygenerowane punkty zostały przedstawione na rys. \ref{fig:test_2}. 
    \begin{figure}[H]
        \centering
        \includegraphics[width=0.6\textwidth]{../gallery/2_test.png}
        \caption{$N = 10^4$ punktów wygenerowanych w każdym kole. $x_\alpha = R_\alpha + R_\beta$.}
        \label{fig:test_2}
    \end{figure}
    
    \noaka Wygenerowane koła są do siebie styczne i nie mają części wspólnej.
    Zgodnie z oczekiwaniami, powierzchnia części wspólnej wynosi $0$. 
    Na podstawie analizy jakościowej, można stwierdzić, że rozkład w kole jest zbliżony do jednorodnego.

    \subsection*{Koła częściowo się przekrywające}

    Wygenerowane koła się przekrywają, tak że wartość parametru $x_\alpha$ została ustawiona na $R_\beta + 0.5 R_\alpha$.
    Milion wygenerowanych pseudolosowo punktów w każdym kole zostało przedstawionych na rys. \ref{fig:ac_circ}.
    \begin{figure}[H]
        \centering
        \includegraphics[width=0.6\textwidth]{../gallery/a_c_circ.png}
        \caption{$N = 10^6$ punktów wygenerowanych w każdym kole. $x_\alpha = R_\beta + 0.5R_\alpha $.}
        \label{fig:ac_circ}
    \end{figure}
    
    \noaka Koła przekrywają się w taki sposób, że szerokość części wspólnej wynosi $0.5R_\alpha$. 
    Prostym wnioskiem jest także fakt, że powierzchnia koła $K_\beta$ jest większa, co skutkuje mniejszą gęstością wylosowanych punktów.
    Dlatego więcej punktów z koła $K_\alpha$ znalazło się w części wspólnej, niż punktów z koła $K_\beta$.
    Wyniki dla obliczeń pola części wspólnej metodą Monte Carlo były zapisywane dla $\{10^k: k \in \{2,3,4,5,6\}\}$ losowań i przedstawione na rys. \ref{fig:ac_mc}.
    \begin{figure}[H]
        \centering
        \includegraphics[width=0.6\textwidth]{../gallery/a_c.png}
        \caption{Wartości pola wspólnego wraz z niepewnościami dla $x_\alpha = R_\beta + 0.5R_\alpha$, gdy: (a) losowane są punkty z koła $K_\alpha$, (c) losowane są punkty z koła $K_\beta$.}
        \label{fig:ac_mc}
    \end{figure}

    \noaka Widoczne jest, że losowanie z koła o większym polu $K_\beta$ generuje większe niepewności i generalnie wolniej zbiega do wartości dokładnej.
    Oczywiście, aby udowodnić tą tezę należałoby wykonać całą serię eksperymentów numerycznych.
    Proste rozumowanie jednak prowadzi do wniosku, że większa gęstość wygenerowanych punktów oznacza większą liczbę punktów w części wspólnej.
    Generując punkty z mniejszego koła istnieje też większe prawdopodobieństwo, że wylosowany punkt będzie należał do części wspólnej.

    \subsection*{Koła całkowicie się przekrywające}

    W drugim przypadku obydwa koła miały ten sam środek ($x_\alpha = 0$), przez co w całości się pokrywają. 
    Koło $K_\alpha$ należy do koła $K_\beta$. Wygenerowane punkty zostały przedstawione na rys. \ref{fig:bd_circ}
    \begin{figure}[H]
        \centering
        \includegraphics[width=0.6\textwidth]{../gallery/b_d_circ.png}
        \caption{$N = 10^6$ punktów wygenerowanych w każdym kole. $x_\alpha =0 $.}
        \label{fig:bd_circ}
    \end{figure}

    \noaka Ponownie można zauważyć, że większa gęstosć punktów (na jednostkę powierzchni) jest w mniejszym kole. 
    Ponadto każdy punkt należący do pola mniejszego, należy także do pola większego.
    W tym przypadku obliczenie pola wspólnego jest proste i polega na obliczeniu pola $K_\alpha$.
    Wyniki obliczeń zostały przedstawione na rys. \ref{fig:bd_mc}.
    \begin{figure}[H]
        \centering
        \includegraphics[width=0.6\textwidth]{../gallery/b_d.png}
        \caption{Wartości pola wspólnego wraz z niepewnościami dla $x_\alpha = 0$, gdy: (b) losowane są punkty z koła $K_\alpha$, (d) losowane są punkty z koła $K_\beta$.}
        \label{fig:bd_mc}
    \end{figure}

    \noaka Prostym wnioskiem jest, że dla losowania punktów z koła $K_\alpha$ uzyskiwana jest od razu wartość dokładna.
    Wynika to z faktu, że funkcja $\theta$ będzie zawsze przyjmowała wartość 1. 
    Przez to wartość sumy w równaniu \eqref{mu_1} będzie równa $N$, a całe wyrażenie będzie zawsze wynosiło $\pi R_\alpha^2$.
    Z kolei wartość drugiego momentu wyrażonego zwykłego wzorem \eqref{mu_2} będzie w takim razie wynosić kwadrat tej wielkości $(\pi R_\alpha^2)^2$.
    Dwa kwadraty się wykasują podczas obliczania odchylenia standardowego zgodnie z \eqref{std_dev}.

    \newpage

    \section{Podsumowanie}

    Na podstawie uzyskanych wyników można postawić tezę, że mniejszy okrąg stanowi lepsze ograniczenie do losowania punktów.
    Przez to większa ilość punktów będzie generowana w części wspólnej, tak jak w metodzie eliminacji.
    Dodatkowo warto się także zastanowić, czy używanie metody Monte Carlo ma sens. 
    W przypadku kół się przekrywających znana jest dokładna odpowiedź.
    Uzyskanie jej w sposób obliczeniowy stanowi jedynie metodę weryfikacji sposobu liczenia.

\end{document}
