\documentclass[a4paper,12pt]{article}

\input{"~/Desktop/Studia/LaTeX/setup.tex"}
\usepackage{svg}

\author{Wojciech Orłowski}
\title{\textsc{Monte Carlo: modelowanie struktury fullerenów metodą
		symulowanego wyżarzania} \\ sprawozdanie}
\date{\today}

\begin{document}
	
	\maketitle
	
	\section*{Wstęp}
	
	Klasyczna dynamika molekularna to dział fizyki (czy chemii) komputerowej zajmujący się badaniem dynamiki cząsteczek za pomocą komputerowych symulacji.
	W takich obliczeniach zakłada się, że atomy/cząsteczki generują pewien znany potencjał, zwykle dobrany tak, aby dla pewnych przypadków dało się odwzorować wyniki doświadczalne.
	Najpopularniejszym sposobem przeprowadzania symulacji klasycznej dynamiki molekularnej jest rozwiązywanie równania ruchu (z drugiej zasady dynamiki Newtona) za pomocą schematu Verleta.
	Takie algorytmy pozwalają też na znajdywanie minimum energetycznego dla ułożeń atomów w pojedynczych cząsteczkach i strukturach periodycznych.
	W ćwiczeniu została użyta metoda symulowanego wyżarzania w celu minimalizacji energii w strukturze fullerenów. 
	
	\subsection*{Metoda symulowanego wyżarzania}
	
	W metodzie tej wybierane jest losowe położenie pojedynczej cząstki. 
	Dla nowego, proponowanego położenia cząstki obliczana jest energia, która jest porównywana z energią poprzedniego ułożenia cząstek.
	Nowe położenie cząstki wybierane jest z prawdopodobieństwem zależnym od różnicy energii - nowej i poprzedniej:
	\begin{equation}
		p_{\textbf{acc}} = \min\left\{1,\exp(- \frac{E_{\text{old}} - E_{\text{new}}}{kT})\right\},
	\end{equation}
	gdzie $kT$ jest stałą Boltzmanna wymnożoną przez temperaturę, czyli w naszym przypadku pewnym parametrem balansującym prawdopodobieństwem akceptacji.
	W ćwiczeniu wprowadzono parametr $\beta = \frac{1}{kT}$, który zmieniał się wraz z iteracjami zgodnie z wzorem
	\begin{equation}
		\beta = \beta_{min} + \left(\frac{it}{it_{max}}\right)^p (\beta_{max} - \beta_{min}),
	\end{equation}
	gdzie $p$ jest parametrem opisującym sposób zmian parametru $\beta$ wraz z iteracjami. 
	Jeżeli $\beta$ jest małe, to występuje duże prawdopodobieństwo, że zostanie zaakceptowane nowe położenie.
	Taki zabieg pozwala opuszczać minima lokalne w wczesnych etapach symulacji, tak aby ostatecznie znaleźć minimum globalne układu.
	
	\subsection*{Potencjał Brennera}
	
	Najważniejszym etapem symulacji klasycznej dynamiki molekularnej jest odpowiedni wybór potencjału.
	W przypadku ćwiczenia wybrany został potencjał Brennera, który może uwzględnić liczbę wiązań tworzonych przez atom.
	Wiemy, że w atomy węgla w fullerenach tworzą maksymalnie 3 wiązania.
	Ponadto potencjał Brennera może dobrze opisywać konfiguracje przestrzenne dla obu hybrydyzacji ($sp^3$ oraz $sp^2$) atomów węgla.
	\\
	\\
	W układzie całkowitą energię potencjalną rozumiemy jako połowę sumy energii wszystkich atomów:
	\begin{equation}
		V_{tot} = \frac{1}{2}\sum_{i=1}^{n} V_i,
	\end{equation}
	przemnożenie przez jedną drugą ma na celu zniwelowanie podwójnego uwzględniania oddziaływania. 
	Potencjał od pojedynczego atomu ma postać sumy oddziaływania z najbliższymi atomami
	\begin{equation}
		V_i = \sum_{\substack{j = 1 \\ j \neq i}}^{n} f_{\text{cut}}(r_{ij})\left[ V_R(r_{ij}) + \overline{B}_{ij}V_A(r_{ij}) \right].
		\label{vi}
	\end{equation}
	W powyższym wyrazie $f_{\text{cut}}$ oznacza funkcję odcięcia, czyli funkcję ograniczającą wagę atomów branych w obliczeniach w zależności od odległości między atomami. Ma ona postać:
	\begin{equation}
		f_{\text{cut}} = 
		\begin{cases}
			1, & r \leq R_1 \\ 
			\frac{1}{2}\left[1 + \cos(\frac{r - R_1}{R_2 - R_1} \pi)\right], & R_1 < r \leq R_2 \\
			0, & r > R_2
		\end{cases}
	\end{equation}
	$R_1, \, R_2$ są parametrami potencjału, a $r$ jest odległością między atomami. Jeżeli odległość między atomami jest większa od $R_2$ (zwanego także promieniem odcięcia) obliczenia dalszych oddziaływań między atomami nie są wykonywane.
	Pozostałe wyrazy w \eqref{vi} odpowiadają za potencjały przyciągania ($V_A$) i odpychania ($V_R$). 
	Opisane są one wzorami:
	\begin{gather}
		V_R(r) = \frac{D_e}{S - 1} \exp(-\sqrt{2S}\lambda\left(r - R_0\right)), \\
		V_A(r) = \frac{D_eS}{S - 1} \exp(-\sqrt{2/S}\lambda\left(r - R_0\right)),
	\end{gather}
	gdzie $D_e, S, \lambda$ oraz $R_0$ są parametrami potencjału.
	Wyraz $\overline{B}$ stanowi pewną wagę potencjału przyciągania. 
	Za pomocą tego potencjału jesteśmy w stanie sterować krotnością wiązania.
	\begin{equation}
		B_{ij} = \left(1 + \zeta_{ij} \right)^{-\delta},
	\end{equation}
	$\delta$ jest parametrem potencjału, a $\zeta_{ij}$ jest zależne bezpośrednio od kąta między wiązaniami
	\begin{gather}
		\zeta_{ij} = \sum_{\substack{k = 1 \\ k \neq i,j}} f_{\text{cut}} g(\theta_{ijk}),\\
		g(\theta_{ijk}) = a_0\left[1 + \frac{c_0^2}{d_0^2} - \frac{c_0^2}{d_0^2 + (1 + \cos(\theta_{ijk}))^2}, \right]
	\end{gather} 
	$a_0$, $c_0$ oraz $d_0$ to parametry potencjału. 
	Dla badanego problemu parametry potencjału zostały ustawione na
\[	
	\begin{array}{|c|c|c|c|c|c|c|c|c|c|}
		\hline
		R_0 [\text{\AA}] & R_1 [\text{\AA}] & R_2 [\text{\AA}] & D_e [\text{eV}] & S & \lambda [\text{\AA}^{-1}] & \delta & a_0 & c_0 & d_0 \\ \hline
		1.315 & 1.70 & 2.00 & 6.325 & 1.29 & 1.5 & 0.080469 & 0.011304 & 19 & 2.5 \\ \hline
	\end{array}
\]	
	\subsection*{Funkcja korelacji par}
	
	W celu zmierzenia działania optymalizacji w sposób ilościowy użyto funkcji korelacji par (pair correlation function - pcf). 
	Określa ona odległość atomów od siebie na histogramie.
	Jest ona liczona za pomocą:
	\begin{gather}
		\Delta r = \frac{2.5 \cdot r_{avg}}{M}\\
		p(m) = \frac{8\pi r_{avg}^2}{n^2} \frac{\sum_{i = 1}^{n} \sum_{j = i+1}^{n} \delta_{m,k}}{2\pi r_m \Delta r} \\ 
		k = floor\left(\frac{r}{\Delta r}\right) 
	\end{gather}
	
	\section*{Wyniki}
	
	\subsection*{Obliczanie energii}
	
	W pierwszym etapie ćwiczenia została obliczona energia struktury podanej przez prowadzącego. 
	Obliczenie tej energii miało na celu sprawdzenie poprawności wykonywanych rachunków, gdyż łatwo byłoby znaleźć w całej procedurze błąd.
	Struktura o optymalnej energii, podana przez prowadzącego została przedstawiona na rys. \ref{test_fulleren}.
	\begin{figure}[h]
		\centering
		\includegraphics[width=0.7\textwidth]{../plots/test_fulleren.png}
		\caption{Zoptymalizowany fulleren; struktura podana przez prowadzącego.}
		\label{test_fulleren}
	\end{figure}
	W takiej strukturze węgiel tworzy wiązania tylko potrójne, tak że ściany fullerenu są sześcio i pięciokątami. 
	Kąty między atomami węgla są zatem równe $120^\circ$.
	Obliczona energia wyniosła:
	\begin{verbatim}
		energia wynosi: -421.622853705101 eV
	\end{verbatim}
	co się zgadza z energią podaną przez prowadzącego. 
	Przy poprawnym obliczaniu energii można przejść do algorytmu symulowanego wyżarzania i optymalizacji struktury.
	
	\subsection*{Zadanie 3}
	
	W celu znalezienia optymalnej struktury fulerenów ustalono parametry optymalizacji:
	
	\[
	\begin{array}{|c|c|c|c|c|c|c|c|c|c|}
		\hline
		n & \beta_{min} & \beta_{max} & p  & it_{max}  & w_r & w_\phi & w_\theta  & W_{all} & r_{start} \left[\text{\AA}\right] \\ \hline
		60 & 1.0 & 100 & 2 & 10^5 & 10^{-4} & 0.05 & 0.05 & 10^{-4} & 3.5 \\ \hline 
 	\end{array}
	\]
	gdzie $w_r, w_\phi, w_\theta, W_{all}$ to wagi odpowiadające za wielkość kroku w odpowiednim kierunku. Struktura początkowa została wyznaczona za pomocą losowych kątów $\theta$ oraz $\phi$ dla każdego atomu, jednak dla stałej odległości od środka układu równej $r_{start}$.
	\begin{figure}[H]
		\centering
		\begin{subfigure}{0.49\textwidth}
			\centering 
			\includegraphics[width=\textwidth]{../plots/task3_random_start.png}
			\caption{}
		\end{subfigure}
		\begin{subfigure}{0.49\textwidth}
			\centering 
			\includegraphics[width=\textwidth]{../plots/task3_after_optim.png}
			\caption{}
		\end{subfigure}
		\caption{(a) Początkowa struktura losowa, niezoptymalizowana, (b) końcowa struktura zoptymalizowana.}
	\end{figure}
	
	\noindent Po zakończeniu obliczeń otrzymano energię:
	\begin{verbatim}
Końcowa energia wynosi: -365.8125596851834 eV
	\end{verbatim}	
	co jest dalekie od wyniku poprawnego (-421.6 eV). 
	Atomy węgla tworzą wiele wiązań, tak, że ściany fulerenów często są trójkątami i kwadratami. 
	W rzeczywistości powinny być one pięciokątami i sześciokątami. 
	Można znaleźć takie atomy węgla, które tworzą 5 wiązań, co jest niezgodne z najbardziej optymalnymi fullerenami.
	W obliczeniach parametr $\beta$ i energia zmieniały się w sposób przedstawiony na wykresie \ref{task3_beta}
	
	\begin{figure}[H]
		\centering
		\includesvg[width=0.5\textwidth]{../plots/task3_beta.svg}
		\caption{Zmiany parametru $\beta$ i energii w trakcie trwania symulacji}
		\label{task3_beta}
	\end{figure}

	\noindent Parametr $\beta$ początkowo wolno rośnie, a pod koniec symulacji szybko osiąga wartość maksymalną.
	Oznacza to, że początkowo atomy będą szybko zmieniały położenia, a pod koniec prawdopodobieństwo zmiany położenia będzie coraz rzadsze.
	Można to zauważyć za pomocą zmiany energii układu - na początku zmienia się ona znacznie, natomiast pod koniec praktycznie się nie zmienia.
	Średnia odległość od początku układów współrzędnych lekko się zmieniła i ostatecznie wyniosła
	\begin{verbatim}
Średnia odległość od środka wyniosła 3.6429550223885765 Å
	\end{verbatim}
	Co jest wartością większą od optymalnej (3.52 \AA).
	Za pomocą funkcji korelacji par jesteśmy w stanie określić promienie kolejnych stref koordynacyjnych.
	Najważniejsza jest jednak pierwsza strefa koordynacyjna, oznaczająca odległość między najbliższymi sąsiadami, czy długość wiązania.
	Na podstawie PCF (rys. \ref{task3_pcf})
	jesteśmy w stanie stwierdzić, że odległość między atomami w pierwszej strefie koordynacyjnej wynosi około 1.460 \AA.
	\begin{figure}[h]
		\centering
		\includesvg[width=0.7\textwidth]{../plots/task3_pcf.svg}
		\caption{Funkcja korelacji par po optymalizacji struktury.}
		\label{task3_pcf}
	\end{figure}
	Pozostałe strefy koordynacyjne są rozmyte, świadczy to o nieregularności struktury wynikającej z optymalizacji średniej jakości.
	
	\subsection*{Zadanie 4}

	W poprzednim zadaniu atomy węgla mogły tworzyć wiązania o krotności większej niż 3 (co jest niezgodne z hybrydyzacją $sp^2$). 
	W celu nałożenia kary na tworzenie się większej ilości wiązań zmieniana jest funkcja $\zeta_{ij}$.
	Jeżeli węgiel tworzy więcej niż 3 wiązania, to kąt między wiązaniami jest mniejszy lub równy 90$^\circ$.
	Dlatego informacja o ilości wiązań zachowana jest w kącie między wiązaniami, a ostateczna postać funkcji $\zeta_{ij}$ to 
	\begin{equation}
		\zeta_{ij} = 
		\begin{cases}
			10, & \cos(\theta_{ijk}) > 0 \\
			\zeta_{ij}^\text{old}, & \text{w przeciwnym wypadku.}
		\end{cases}
	\end{equation}
	Czyli kara ma postać zwrócenia dużej wartości, gdy kąt między wiązaniami jest mały.
	Dla tych samych wartości parametrów co w zadaniu 3, z zmienioną funkcją $\zeta_{ij}$ struktura początkowa i zoptymalizowana zostały przedstawione na rys. \ref{task4_optim}
	\begin{figure}[H]
		\centering
		\begin{subfigure}{0.49\textwidth}
			\centering 
			\includegraphics[width=\textwidth]{../plots/task4_random_start.png}
			\caption{}
		\end{subfigure}
		\begin{subfigure}{0.49\textwidth}
			\centering 
			\includegraphics[width=\textwidth]{../plots/task4_after_optim.png}
			\caption{}
		\end{subfigure}
		\caption{(a) Początkowa struktura losowa, niezoptymalizowana, (b) końcowa struktura zoptymalizowana.}
		\label{task4_optim}
	\end{figure}
	
	\noindent Końcowa energia oraz końcowy średni promień od środka układ współrzędnych wyniosły
	\begin{verbatim}
		Końcowa energia wynosi: -410.83556770359144 eV
		Średnia odległość od środka wyniosła 3.622191266744983 Å
	\end{verbatim}
	Końcowa energia układu jest dużo bliższa energii faktycznej zoptymalizowanej struktury. 
	Widoczne są wciąż nieregularności w strukturze, jednak tworzy ona pięcio i sześciokąty.
	Końcowa odległość od układu początku układu współrzędnych również jest lekko za wysoka. 
	Wykres zmian energii układu i parametru $\beta$ został przedstawiony na rys. \ref{task4_beta}.
	\begin{figure}[H]
		\centering
		\includesvg[width=0.7\textwidth]{../plots/task4_beta.svg}
		\caption{Zmiany parametru $\beta$ i energii w trakcie trwania symulacji}
		\label{task4_beta}
	\end{figure}
	
	\noindent Zauważalne są mniejsze oscylacje energii układu. 
	Szybciej ona osiąga mniejsze wartości.
	Energia też cały czas jest zmniejszana, nawet pod koniec symulacji, gdzie $\beta$ jest duże.
	Zmiany te zachodzą jednak dużo wolniej. 
	Z wykresu pcf (rys. \ref{task4_pcf}) jesteśmy w stanie odczytać promienie stref koordynacyjnych.
	\begin{figure}[h]
		\centering
		\includesvg[width=0.7\textwidth]{../plots/task4_pcf.svg}
		\caption{Funkcja korelacji par po optymalizacji struktury.}
		\label{task4_pcf}
	\end{figure}
	
	\noindent Z wykresu można odczytać, że długość wiązania wyniosła 1.457 \AA, czyli podobnie jak w poprzednich obliczeniach. 
	Dużo widoczniejsze są kolejne strefy koordynacyjne, w których możemy wyróżnić promienie 2.37 \AA (druga strefa koordynacyjna) oraz 3.64 \AA (trzecia strefa koordynacyjna).
	Wyraźniejsze piki potwierdzają lepsze, bardziej regularne ułożenie struktury.
	W strukturze jednak zdarzają się nieregularności.
	Przykładowa nieregularność została przedstawiona na rys. \ref{task4_niedobrze}.
	\begin{figure}[h]
		\centering
		\includegraphics[width=0.5\textwidth]{../plots/task4_niereg.png}
		\caption{Przykładowa nieregularność w strukturze}
		\label{task4_niedobrze}
	\end{figure}
	Widoczny jest nieregularny wielokąt, który nie znajduje się w optymalnej strukturze fullerenów.
	
	\subsection*{Zadanie 5}
	
	Widoczne są duże zmiany w promieniu początkowym i końcowym, więc propozycją jest wprowadzenie zaburzenia w postaci początkowego promienia ułożenia atomów.
	Atomy węgla początkowo zostały rozłożone na sferze o promieniu 2.5 \AA.
	Struktura początkowa i po optymalizacji zostały przedstawione na rys. \ref{task5_optim}
	\begin{figure}[H]
		\centering
		\begin{subfigure}{0.49\textwidth}
			\centering 
			\includegraphics[width=\textwidth]{../plots/task5_random_start.png}
			\caption{}
		\end{subfigure}
		\begin{subfigure}{0.49\textwidth}
			\centering 
			\includegraphics[width=\textwidth]{../plots/task5_after_optim.png}
			\caption{}
		\end{subfigure}
		\caption{(a) Początkowa struktura losowa, niezoptymalizowana, (b) końcowa struktura zoptymalizowana.}
		\label{task5_optim}
	\end{figure}
	
	\noindent Pierwszym wnioskiem wynikającym z optymalizacji jest zmiana promienia fullerenu. 
	Zmiana tego promienia w trakcie trwania symulacji została przedstawiona na rys. 
	\ref{task5_r}.
	\begin{figure}[h]
		\centering
		\includesvg[width=0.6\textwidth]{../plots/task5_r.svg}
		\caption{Promień fullerenu w zależności od iteracji.}
		\label{task5_r}
	\end{figure}
	
	\noindent Promień w początkowych czasach symulacji szybko rośnie, a następnie się stabilizuje.
	Po stabilizacji średni promień oscyluje wokół wartości 3.61 \AA. 
	Jest to ta sama (podobna) wartość, jaką otrzymano zaczynając obliczenia od początkowej odległości 3.5 \AA.
	Otrzymana końcowa energia i wartość średniej odległości od środka układu wyniosły
	\begin{verbatim}
Końcowa energia wynosi: -408.3394305712258 eV
Średnia odległość od środka wyniosła: 3.611961520829443 Å
	\end{verbatim}
	Energia jest trochę mniejsza od wcześniej uzyskiwanych energii, jednak jej ostateczna wartość zależy także od ułożenia początkowego, które jest losowe. 
	Zmiany energii wraz z zmianami parametru $\beta$ zostały przedstawione na rys. \ref{task5_beta}.
	\begin{figure}[H]
		\centering
		\includesvg[width=0.7\textwidth]{../plots/task5_beta.svg}
		\caption{Zmiany parametru $\beta$ i energii w trakcie trwania symulacji.}
		\label{task5_beta}
	\end{figure}	
	
	\noindent Widoczna duża zmiana energii następuje dopiero w około 25000 iteracji. 
	Wcześniej energia maleje z małym tempem.
	Na wykresie funkcji korelacji par dla struktury zoptymalizowanej (rys. \ref{task5_pcf}) można odczytać strefy koordynacyjne.
	\begin{figure}[H]
		\centering
		\includesvg[width=0.7\textwidth]{../plots/task5_pcf.svg}
		\caption{Funkcja korelacji par po optymalizacji struktury.}
		\label{task5_pcf}
	\end{figure}
	
	\noindent Z wykresu pcf można odczytać wartości nawet 5 promieni stref koordynacyjnych.
	Pierwsza strefa koordynacyjna jest bardzo wyraźna, a pozostałe co raz bardziej rozmyte, jednak wciąż rozróżnialne od otoczenia.
	
	\subsection*{Zadanie 6}
	
	Wcześniej przedstawiony zestaw parametrów można zmodyfikować i przeanalizować wpływ niektórych parametrów na przebieg obliczeń. 
	Można zmienić parametry dotyczące zmian $\beta$ oraz te dotyczące ruchów atomów.  
	
	\subsubsection*{Zmiana przebiegu $\beta$}
	
	W poprzednim przykładzie $\beta$ zmieniała się z wykładnikiem $p = 2$. 
	Przez to w układzie początkowo była \textit{wysoka temperatura}, a cząstki miały większe prawdopodobieństwo przeskoku. 
	Można sprawdzić, czy symulacja będzie stabilniejsza dla $p < 1$.
	Wynik końcowy optymalizacji z $p = 0.5$ zwrócił położenie atomów o energii
	\begin{verbatim}
		Końcowa energia wynosi: -410.54807531075414 eV,
	\end{verbatim}
	Struktura otrzymana ostatecznie została przedstawiona na rys. \ref{task6_optim}
		\begin{figure}[h]
		\centering
		\includegraphics[width=0.7\textwidth]{../plots/task6_try_1.png}
		\caption{Zoptymalizowany fulleren dla $p = 0.5$}
		\label{task6_optim}
	\end{figure}
	
	\noindent Zoptymalizowana struktura ma nieregularności.
	Jednak ma też dużo cech charakterystycznych dla tego typu struktur - dużo pięciokątów i sześciokątów.
	Przebieg zmian energii podczas optymalizacji został przedstawiony na rys. \ref{task6_beta}
	
	\begin{figure}[H]
		\centering
		\includesvg[width=0.7\textwidth]{../plots/task6_beta_1.svg}
		\caption{Zmiany parametru $\beta$ i energii w trakcie trwania symulacji.}
		\label{task6_beta}
	\end{figure}
	
	\noindent Szybko uzyskiwana jest zbieżność, lecz później energie w symulacji zmieniają się bardzo powoli.
	Końcowy efekt jest bardzo podobny do uzyskiwanych wcześniej.
	Można zauważyć mniejsze oscylacje energii, przebieg wygląda stabilniej.
	
	\subsubsection*{Zmiany parametrów $w$}
	
	Zamiast zmieniać sposobu przebiegu parametru $\beta$ można spróbować zmienić wagi zmiany położeń atomów.
	Zostały zmienione one na:
	\[
	\begin{array}{|c|c|c|c|}
		\hline 
		w_r & w_\theta & w_\phi & w_{all} \\ \hline
		5.0\cdot 10^{-5} & 1.00 & 1.00 & 10^{-3} \\ \hline
	\end{array}
	\]
	Zwiększone zostały wagi po kątach $\theta$ oraz $\phi$, natomiast zmniejszona została waga kroku po $r$. 
	Aby zrekompensować zmianę kroku po $r$ zwiększona została waga zmiany promienia przy próbie zmiany promienia całej struktury - $w_{all}$.
	
	\begin{figure}[H]
		\centering
		\includegraphics[width=0.6\textwidth]{../plots/task6_try_2.png}
		\caption{Zoptymalizowany fulleren dla zmienionych wag ruchów atomów}
		\label{task6_optim_2}
	\end{figure}
	
	\noindent Uzyskana struktura ma promień podobny do innych struktur. 
	Cała składa się z wielokątów, które nie wyglądają na regularne.
	Wykres zmian energii i parametru $\beta$ został przedstawiony na rys. \ref{task6_beta2}
	
	\begin{figure}[H]
		\centering
		\includesvg[width=0.7\textwidth]{../plots/task6_beta_2.svg}
		\caption{Zmiany parametru $\beta$ i energii w trakcie trwania symulacji.}
		\label{task6_beta2}
	\end{figure}
	
	\noaka W badanym przypadku uzyskano bardzo szybko zbieżność. 
	Niemal natychmiastowo.
	Po osiągnięciu energetycznego minimum energia powoli malała liniowo.
	Jest to związane z szybkim ustaleniem odpowiedniego promienia. 
	Zmiany promienia (średniej odległości od środka układu) zostały przedstawione na rys. \ref{task6_r}.
	\begin{figure}[h]
		\centering
		\includesvg[width=0.6\textwidth]{../plots/task6_r_2.svg}
		\caption{Promień fullerenu w zależności od iteracji.}
		\label{task6_r}
	\end{figure}
	
	\noindent Promień cząsteczki jest trochę większy po stabilizacji.
	Większy promień generuje kilka dużych wielokątów wewnątrz cząsteczki, które z czasem rozrastają się w dziury.
	
	\subsubsection*{Podsumowanie zmiany wpływu parametrów}
	
	Sterowanie parametrami symulacji jest ciężkie i niełatwo jest znaleźć optymalny ich zestaw.
	Większość kombinacji daje podobny wynik końcowy, jednak ciężko je porównać, gdyż energia optymalna jest też zmienną losową.
	W zależności od początkowego ułożenia atomów a także samego przebiegu dostajemy różne rezultaty.
	W celu znalezienia najlepszego zestawu trzeba wykonać dużą ilość symulacji dla pojedynczego zestawu parametrów. 
	
	\subsection*{Zadanie 7}
	
	Wykonane zostały optymalizacje struktury fullerenów dla $n = [30,31,\dots,60]$ atomów węgla.
	W trakcie trwania symulacji zmienia się energia pojedynczego wiązania oraz odległość atomów od środka układu współrzędnych. Zoptymalizowane struktury wraz z rosnącą liczbą atomów w symulacji zostały przedstawione na rys. \ref{task7_optim}.
	To co można bezpośrednio zauważyć na podstawie rysunków struktur po optymalizacji to zmiana promieni fullerenów.
	Na każdej wizualizacji z rys. \ref{task7_optim} została przedstawiona struktura, wraz z sferą dla referencji. 
	Sfera ma stały promień równy 2.4 \AA. 
	Widać zatem, że dla mniejszej ilości atomów atomy układały się bliżej tej sfery.
	Zależność otrzymanej średniej odległości od środka układów współrzędnych zostały przedstawione na rys. \ref{task7_r}(a).
	\begin{figure}[h]
		\centering
		\begin{subfigure}{0.49\textwidth}
			\centering
			\includesvg[width=\textwidth]{../plots/rb.svg}
			\caption{}
		\end{subfigure}
		\begin{subfigure}{0.49\textwidth}
			\centering
			\includesvg[width=\textwidth]{../plots/eb.svg}
			\caption{}
		\end{subfigure}
		\caption{(a) Promień fullerenu w zależności od ilości atomów, (b) energia wiązania przypadająca na jeden atom w strukturze w zależności od ilości atomów}
		\label{task7_r}
	\end{figure}
	Promień zwiększa się liniowo wraz z wzrostem ilości atomów. 
	Dodatkowe atomy w strukturze wpływają także na jej energie. 
	Co raz większa liczba wiązań ma odzwierciedlenie w energii pojedynczego wiązania.
	Energia wiązań przypadająca pojedynczemu atomowi została przedstawiona na rys. \ref{task7_r}(b).
	Energia wiązania jest zdecydowanie niższa dla struktury o większej ilości atomów.
	Jednak ciężko jest dokładnie te energie określić, wiedząc że sama optymalizacja nie jest stabilna najlepiej i wymaga dobrej parametryzacji, co widać po dużych oscylacjach wartości energii.

	\begin{figure}[H] 
		\centering
		\captionsetup[subfigure]{labelformat=empty}
		\begin{subfigure}{0.19\textwidth}
			\centering 
			\includegraphics[width=\textwidth]{../plots/task7/cropped_30.png}
			\caption{n = 30}
		\end{subfigure}
		\begin{subfigure}{0.19\textwidth}
			\centering 
			\includegraphics[width=\textwidth]{../plots/task7/cropped_32.png}
			\caption{n = 32}
		\end{subfigure}
		\begin{subfigure}{0.19\textwidth}
			\centering 
			\includegraphics[width=\textwidth]{../plots/task7/cropped_34.png}
			\caption{n = 34}
		\end{subfigure}
		\begin{subfigure}{0.19\textwidth}
			\centering 
			\includegraphics[width=\textwidth]{../plots/task7/cropped_36.png}
			\caption{n = 36}
		\end{subfigure}
		\begin{subfigure}{0.19\textwidth}
			\centering 
			\includegraphics[width=\textwidth]{../plots/task7/cropped_38.png}
			\caption{n = 38}
		\end{subfigure}
		\\ 
		\begin{subfigure}{0.19\textwidth}
			\centering 
			\includegraphics[width=\textwidth]{../plots/task7/cropped_40.png}
			\caption{n = 40}
		\end{subfigure}
		\begin{subfigure}{0.19\textwidth}
			\centering 
			\includegraphics[width=\textwidth]{../plots/task7/cropped_42.png}
			\caption{n = 42}
		\end{subfigure}
		\begin{subfigure}{0.19\textwidth}
			\centering 
			\includegraphics[width=\textwidth]{../plots/task7/cropped_44.png}
			\caption{n = 44}
		\end{subfigure}
		\begin{subfigure}{0.19\textwidth}
			\centering 
			\includegraphics[width=\textwidth]{../plots/task7/cropped_46.png}
			\caption{n = 46}
		\end{subfigure}
		\begin{subfigure}{0.19\textwidth}
			\centering 
			\includegraphics[width=\textwidth]{../plots/task7/cropped_48.png}
			\caption{n = 48}
		\end{subfigure}
		\\ 
		\begin{subfigure}{0.19\textwidth}
			\centering 
			\includegraphics[width=\textwidth]{../plots/task7/cropped_50.png}
			\caption{n = 50}
		\end{subfigure}
		\begin{subfigure}{0.19\textwidth}
			\centering 
			\includegraphics[width=\textwidth]{../plots/task7/cropped_52.png}
			\caption{n = 52}
		\end{subfigure}
		\begin{subfigure}{0.19\textwidth}
			\centering 
			\includegraphics[width=\textwidth]{../plots/task7/cropped_54.png}
			\caption{n = 54}
		\end{subfigure}
		\begin{subfigure}{0.19\textwidth}
			\centering 
			\includegraphics[width=\textwidth]{../plots/task7/cropped_56.png}
			\caption{n = 56}
		\end{subfigure}
		\begin{subfigure}{0.19\textwidth}
			\centering 
			\includegraphics[width=\textwidth]{../plots/task7/cropped_58.png}
			\caption{n = 58}
		\end{subfigure}
		\\ 
		\begin{subfigure}{0.19\textwidth}
			\centering 
			\includegraphics[width=\textwidth]{../plots/task7/cropped_60.png}
			\caption{n = 60}
		\end{subfigure}
		\caption{Zoptymalizowane struktury o różnych ilościach atomów.}
		\label{task7_optim}
	\end{figure}
	
	\newpage
	
	\subsection*{Kontrola przebiegu optymalizacji}
	
	Ważnym elementem optymalizacji jest także kontrola jej przebiegu.
	W dłuższych, dokładniejszych obliczeniach błędy numeryczne mogą doprowadzić do rozerwania struktury, co by wymagało powtórzenia obliczeń.
	Ważnym parametrem jakim można śledzić podczas optymalizacji jest funkcja korelacji par (pcf).
	Dla szybkiej, testowej optymalizacji (10000 iteracji) rozpoczynającej od losowego położenia atomów w promieniu 2.5 \AA $\,$ od środka układu współrzędnych dokonano zapisu struktury w kilku momentach.
	Funkcje korelacji par dla $it = 1,10,100,1000$ zostały przedstawione na rys. \ref{optim_pcf}.
	\begin{figure}[h]
		\begin{subfigure}{0.49\textwidth}
			\centering
			\includesvg[width=\textwidth]{../plots/task_add_pcf_it1.svg}
			\caption{}
		\end{subfigure}
		\begin{subfigure}{0.49\textwidth}
			\centering
			\includesvg[width=\textwidth]{../plots/task_add_pcf_it10.svg}
			\caption{}
		\end{subfigure}
		\\
		\begin{subfigure}{0.49\textwidth}
			\centering
			\includesvg[width=\textwidth]{../plots/task_add_pcf_it100.svg}
			\caption{}
		\end{subfigure}
		\begin{subfigure}{0.49\textwidth}
			\centering
			\includesvg[width=\textwidth]{../plots/task_add_pcf_it1000.svg}
			\caption{}
		\end{subfigure}
		\caption{Funkcje korelacji par dla (a) pierwszej iteracji, (b) dziesiątej iteracji, (c) setnej iteracji, (d) tysięcznej iteracji.}
		\label{optim_pcf}
	\end{figure}
	
	\noindent na podstawie pcf można zauważyć w którym momencie symulacji atomy zaczęły się układać w regularną strukturę.
	Dla pierwszej i dziesiątej iteracji widać dużą ilość atomów zebranych w klastry, o małych odległościach wokół siebie.
	W kolejnych iteracjach widać, że te odległości się zwiększają, a dla tysięcznej można zauważyć kolejne strefy koordynacyjne.
	Ułożenia atomów odpowiadające przedstawionym PCF zostały przedstawione na rys. \ref{optim_it}.
	\begin{figure}[H]
		\begin{subfigure}{0.49\textwidth}
			\centering
			\includegraphics[width=\textwidth]{../plots/task_add_it1.png}
			\caption{}
		\end{subfigure}
		\begin{subfigure}{0.49\textwidth}
			\centering
			\includegraphics[width=\textwidth]{../plots/task_add_it10.png}
			\caption{}
		\end{subfigure}
		\\
		\begin{subfigure}{0.49\textwidth}
			\centering
			\includegraphics[width=\textwidth]{../plots/task_add_it100.png}
			\caption{}
		\end{subfigure}
		\begin{subfigure}{0.49\textwidth}
			\centering
			\includegraphics[width=\textwidth]{../plots/task_add_it1000.png}
			\caption{}
		\end{subfigure}
		\caption{Ułożenia atomów w (a) pierwszej iteracji, (b) dziesiątej iteracji, (c) setnej iteracji, (d) tysięcznej iteracji.}
		\label{optim_it}
	\end{figure}
	
	\noindent Ciekawym aspektem jest ułożenie się atomów w kształty \textit{niedozwolone} czyli takie, które wymagają się od atomów tworzenia więcej niż 4 wiązań.
	Taka struktura jest bardzo regularna, lecz nierzeczywista. 
	Wynika to z nałożenia małego promienia fullerenu na początku symulacji.
	Przez niską ilość iteracji atomy nie zdążyły się rozszerzyć i ułożyły się w niedozwolone wielokąty.
	
	
	\newpage
	
	\section*{Podsumowanie}
	
	Dynamika molekularna jest bardzo popularnym działem nauki, dlatego można skorzystać z wielu już napisanych narzędzi.
	Przykładowo są to programy do budowania i wizualizacji cząstek takie jak \textbf{IQMol} czy \textbf{Avogadro}. 
	Przykładowe wizualizacje otrzymanych fullerenów zostały przedstawione na rys. \ref{pods_8}.
	
	\begin{figure}[H]
		\begin{subfigure}{0.49\textwidth}
			\centering
			\includegraphics[width=\textwidth]{../plots/iqmol.png}
			\caption{}
		\end{subfigure}
		\begin{subfigure}{0.49\textwidth}
			\centering
			\includegraphics[width=\textwidth]{../plots/avogadro.png}
			\caption{}
		\end{subfigure}
		\caption{Wizualizacje zoptymalizowanej struktury w (a) IQMol oraz (b) Avogadro.}
		\label{pods_8}
	\end{figure}
	
	\noaka Programy do wizualizacji często mają opcje stworzenia pliku wejściowego do bardziej zaawansowanego programu obliczeniowego, wykorzystującego teorię DFT (kwantowa dynamika molekularna).
	Jak można zauważyć na rys. \ref{pods_8}(a) \textbf{IQMol} jest w stanie samodzielnie znaleźć podwójne wiązania między atomami węgla, co stanowi wartościową interpretacje mającą odwzorowanie w teorii hybrydyzacji.
	\\
	\\
	Prosty algorytm mający na celu zoptymalizowanie struktury nie uzyskał (dla żadnych parametrów) struktury idealnej, przedstawionej na rysunku \ref{test_fulleren}.
	Uzyskanie takiej struktury wymagało by bardzo długiego, kontrolowanego \textit{wyżarzania} lub wykorzystania innego algorytmu i teorii.
	Warto jednak zwrócić uwagę na obliczone energie struktur zoptymalizowanych.
	Są one bardzo bliskie energii idealnej, różnią się zaledwie o kilkanaście elektronowoltów.
	\\
	\\
	Obliczenia i optymalizacja algorytmu zostały wykonane w języku \julia $\,$ służącym między innymi do obliczeń numerycznych.
	Wizualizacja struktur odbyła się za pomocą biblioteki \texttt{Makie} z interaktywnym silnikiem \texttt{GLMakie}.
	
\end{document}



