\documentclass[a4paper, 12pt]{article}

\input{~/Desktop/Studia/LaTeX/setup.tex}
\usepackage{amsmath}


\author{Wojciech Orłowski}
\date{\today}
\title{\textsc{Monte Carlo: symulacja rozpraszania fotonów w ośrodku częściowo
		przeźroczystym}\\ - sprawozdanie}


\definecolor{LightCyan}{rgb}{0.88,1,1}
	
\begin{document}
	\maketitle
	
	\section*{Wstęp}
	
	Podstawowym zagadnieniem elektrodynamiki jest propagacja fali elektromagnetycznej.
	W klasycznym przypadku opis propagacji fali elektromagnetycznej opisuje układ równań Maxwella- czterech równań różniczkowych cząstkowych. 
	Rozwiązywanie tego układu równań jest ciężkie obliczeniowo, a zwykle się wykonuje za pomocą odpowiedniej dyskretyzacji w czasie i skorzystania z metody różnic skończonych (FDTD - finite differences time domain).
	Na symulowany układ nakładane są pewne makroskopowe wielkości (zespolona stała dielektryczna, magnetyczna itp.), wynikające z struktury materiału. 
	Metody te jednak nie radzą sobie najlepiej z ośrodkami mętnymi, w których następuję odbicie od cząsteczek zawieszonych w zawiesinie oraz są bardzo wymagające obliczeniowo.
	Ciekawą alternatywą jest wykorzystanie metody Monte Carlo propagacji fotonów.
	\\
	\\
	Traktując zagadnienie propagacji światła jako zagadnienie poruszania się fotonów (nie oddziałujących z sobą) można interpretować absorpcję i rozpraszanie jako zjawiska stochastyczne.
	W tym celu trzeba zdefiniować sposób działania propagacji fotonów w ośrodku
	Dla makroskopowych współczynników takich jak współczynniki absorpcji ($\mu_a$) oraz współczynniki rozpraszania ($\mu_s$) definiuje się całkowity współczynnik oddziaływania z układem.
	\[ \mu_{tot} = \mu_a + \mu_s. \]
	Prawdopodobieństwo, że foton nie ulegnie interakcji z układem jest większe dla mniejszego współczynnika $\mu_{tot}$. 
	Dlatego określamy prawdopodobieństwo, że na drodze $s$ nie dojdzie do rozproszenia/absorpcji jako
	\[ P(x < s) = \exp(-\mu_{tot}\cdot s). \] 
	Można rozponać danę prawdopobieństwo jako dystrybuantę rozkładu eksponencjalnego, z czego wynika, że rozkład drogi swobodnej jest zadany przez wyrażenie:
	\[ s = -\frac{\ln(U_1)}{\mu_{tot}}, \; \; \; U1 \sim U(0,1)\] 
	Po propagacji fali w konkretnym kierunku należy rozważyć efekty absorpcji i rozpraszania.
	W początkowej wiązce istnieje współczynnik określający frakcję istniejącej cząstki (na początku wynosi po prostu $w = 1$).
	Po absorpcji i rozproszeniu z początkowej frakcji cząstki usuwana jest pewna jej część.
	Pozostała część wiązki jest propagowana w zmienionym kierunku.
	Część wiązki jaka jest absorbowana jest obliczana na podstawie parametrów układu:
	\[ 
		\Delta w = \frac{\mu_a}{\mu_a + \mu_s};
		 \; \; \; 
		 w \rightarrow w - \Delta w 	
	\] 
	Dodatkowo w dwuwymiarowej tablicy przechowywana jest informacja o ilościach zaabsorbowanej wiązki w danym miejscu.
	Po absorpcji wiązki jej pozostała część zmienia kierunek propagacji.
	Aby ustalić w jaki sposób wiązka zmienia kierunek propagacji skorzystano z rozkładu rozproszenia Henyey’a-Greenstein’a: 
	\[ f(\theta) =  \frac{1}{4\pi} \frac{1 - g^2}{(1+g^2 - 2g \cos\theta)^{3/2}}, \]
	gdzie $g$ jest pewnym makroskopowym współczynnikiem nazywanym współczynnikiem anizotropii. 
	Pozwala on określić losową wartość kosinusa kąta rozpraszania za pomocą wzorów:
	\begin{equation}
		\cos\theta = 
		\begin{cases}
			\frac{1}{2g}\left[1 + g^2 - \left(\frac{1-g^2}{1 - g + 2gU_1}\right)\right], \; \; U_1 \sim U(0,1) & g \neq 0, \\ 
			2U_1 - 1, \; \; U_1 \sim U(0,1) & g = 0.
		\end{cases}
	\end{equation}
	Ostatnim aspektem jaki jest brany pod uwagę są zjawiska na granicy ośrodków. 
	Gdy wiązka przechodzi przez granicę ośrodków należy sprawdzić zjawiska takie jak:
	\begin{itemize}
		\item Całkowite wewnętrzne odbicie.
		\\ Występuje dla współczynników załamania światła $n_i$, $n_s$ oraz kąta padania większemu od kąta krytycznego 
		\begin{equation}
			\alpha_i^{\text{critical}} = \arcsin\frac{n_s}{n_i}.
		\end{equation}
		Wówczas wiązka pozostaje całkowicie odbita i jej kierunek propagacji w kierunku $y$ zostaje obrócony.
		\item Odbicie i przejście przez granicę faz.
		\\
		Zgodnie z prawem Snella znając współczynniki załamania światła dwóch materiałów tworzących interfejs ($n_i,n_s$) oraz kąt padania wiązki ($\alpha_i$) jesteśmy w stanie określić kąt pod którym wiązka przechodzi ($\alpha_s$)
		\[n_i \sin(\alpha_i) = n_s\sin(\alpha_s). \]
		Wiemy także, że wiązka odbija się pod takim samym kątem, pod jakim pada.
		W celu obliczenia części wiązki, jaka ulega odbiciu obliczany jest współczynnik odbicia z wzorów Fresnela \eqref{fresnel}.
		\begin{equation}
			R_s = \left|\frac{n_1\cos\theta_i - n_2\cos\theta_t}{n_1\cos\theta_i + n_2\cos\theta_t}\right|^2, \; \; \; ; \; \; \; 
			R_p = \left|\frac{n_1\cos\theta_t - n_2\cos\theta_i}{n_1\cos\theta_t + n_2\cos\theta_i}\right|^2,
			\label{fresnel}
		\end{equation} 
		gdzie $R_s$ i $R_p$ to współczynniki odbicia dla wiązki o polaryzacji $s$ (TE - transverse electric) i $p$ (TM - transverse magnetic). 
		W badanym przypadku nie korzystamy z wiązki spolaryzowanej, dlatego została wzięta średnia z dwóch wyrażeń
		\begin{equation}
			R = \frac{1}{2} \left[\frac{\sin^2(\alpha_i-\alpha_s)}{\sin^2(\alpha_i+\alpha_s)} + \frac{\tan^2(\alpha_i-\alpha_s)}{\tan^2(\alpha_i+\alpha_s)}\right]
		\end{equation}
		W symulacji nie dzielono wiązki, tylko przejście jest uzależnione w sposób losowy, co po uśrednieniu da taki sam efekt:
		\begin{equation}
			U_1 \sim U(0,1) \rightarrow \begin{cases}
				U_1 \leq R  & - \text{odbicie wiązki},\\
				U_1 > R & - \text{wiązka w całości przechodzi}.
			\end{cases}
		\end{equation}
		\item Kasowanie wiązki
		\\ 
		Aby nie tracić czasu na propagację wiązki o bardzo małej frakcji (wadze) jest ona traktowana schematem ruletki - jest ona kasowana lub wzmacniana. 
		\begin{verbatim}
			if(w < 10^-4){
				p_amp = 0.1
				U1 = rand()
				if(U1 <= p_amp){
					w = w / p_amp // (wzmocnienie cząstki)
				}else{
					w = 0 // (kasowanie cząstki)
				}
			}
		\end{verbatim}
	\end{itemize}
	
	\noindent Całość ćwiczenia została wykonana z użyciem przygotowanej klasy \texttt{PHOTON$\_$DIFFUSION$\_$2D} implementującej przedstawiony algorytm.

	\section*{Wyniki}
	
	Podczas ćwiczenia przetestowane zostało działanie algorytmu dla kilku parametrów warstw w układzie trójwarstwowym. 
	Podstawowe parametry symulacji zostały przedstawione w tabeli \ref{tab:params}.
	
	\begin{table}[H]
	\centering
	\caption{Podstawowe parametry symulacji.}
	\label{tab:params}
		\begin{tabular}{|c|c|c|c|c|c|c|c|c|}
			\hline
			\rowcolor{LightCyan}
			$nlayers$ & $x\_max$ & $x\_source$ & $dx\_source$ & $x\_detect$ & $dx\_detect$ & $nx/ny$ & $rx/ry$ & $N$ \\ \hline
			3 & 0.2 & 0.1 & 0.0 & 0.15 & 0.01 & 100 & 0.0 & 200000 \\   
			\hline  
		\end{tabular}
	\end{table}
	
	\noindent Natomiast podstawowe parametry materiałowe warstw zostały przedstawione w tabeli \ref{tab:lay_params}
	\begin{table}[H]
		\centering
		\caption{Podstawowe parametry materiałowe każdej z warstw.}
		\label{tab:lay_params}
		\begin{tabular}{|c|c|c|c|c|c|}
			\hline
			\rowcolor{LightCyan}
			$m$ & $\mu_a$ [1/cm] & $mu_s$ [1/cm] & $d$ [cm] & g [-] & n[-] \\ \hline
			1 & 1 & 10 & 0.02 & 0.75 & 1.3 \\ \hline
			2 & 1 & 190 & 0.02 & 0.075 & 1.0 \\ \hline
			3 & 10 & 90 & 0.02 & 0.95 & 1.0 \\
			\hline  
		\end{tabular}
	\end{table}
	
	\noindent Przedstawione parametry będą się zmieniać w zależności od przeprowadzanych obliczeń.
	
	
	\subsection*{Odbicia wewnętrzne}
	
	W pierwszych obliczeniach ustawiono $rx = 0.8$ oraz $ry = 0.6$ w celu zaobserwowania odbić od kolejnych warstw. 
	Dla kolejnych obliczeń zmieniano parametry warstw zgodnie z tabelą \ref{tab:task2}
	
	\begin{table}[H]
		\caption{Zmiany parametrów warstw w kolejnych obliczeniach.}
		\label{tab:task2}
		\centering
		\begin{tabular}{|c|c|c|c|}
			\hline
			\rowcolor{LightCyan}
			nr. & $n_1$ & $n_2$ & $\mu_{s2}$ \\ \hline
			1 	& -		& 1.5	& - \\ \hline
			2	& - 	& 2.5	& - \\ \hline
			3	& 1.0 	& 1.5	& - \\ \hline
			4 	& 1.0	& 1.5	& 10 \\ \hline
		\end{tabular}
	\end{table}
	
	\noindent Wyniki obliczeń zostały przedstawione na rys. \ref{fig:task2_calc}.
	
	\begin{figure}[H]
		\centering
		\begin{subfigure}{0.49\textwidth}
			\centering
			\includegraphics[width=\textwidth]{../plots/task2_0.pdf}
			\caption{}
		\end{subfigure}
		\begin{subfigure}{0.49\textwidth}
			\centering
			\includegraphics[width=\textwidth]{../plots/task2_1.pdf}
			\caption{}
		\end{subfigure}
		\\
				\begin{subfigure}{0.49\textwidth}
			\centering
			\includegraphics[width=\textwidth]{../plots/task2_2.pdf}
			\caption{}
		\end{subfigure}
		\begin{subfigure}{0.49\textwidth}
			\centering
			\includegraphics[width=\textwidth]{../plots/task2_3.pdf}
			\caption{}
		\end{subfigure}
	\caption{Wyniki propagacji wiązki fotonów dla parametrów podstawowych z zmianami z tabeli \ref{tab:task2}: (a) $n_2$ = 1.5, (b) $n_2$ = 2.5, (c) $n_1$ = 1.0, $n_2$ = 1.5 (d) $n_1$ = 1.0, $n_2$ = 1.5, $\mu_{s2}$ = 10. Na każdym z wykresów przedstawione: u góry - skumulowana transmisja, na dole - skumulowane odbicie, w środku - skumulowana absorpcja podczas propagacji.}
	\label{fig:task2_calc}
	\end{figure}
	 
	\noindent Analizując powyżej przedstawione układy można stwierdzić ważne wnioski.
	W pierwszym przypadku (rys. \ref{fig:task2_calc}(a)) różnica współczynników załamania światła była zbyt mała i nie zaszło całkowite wewnętrzne odbicie w żadnym z przypadków.
	Duże i wyraźne odbicia możemy zauważyć w przypadkach, gdzie różnica między współczynnikami była największa - (b), (c), (d). 
	W przypadku (a) duża część światła rozproszyła się wstecz, jednak nie zauważono z pomiarów absorpcji dużej wiązki odbitej.  
	Dla obliczeń przedstawionych w podpunktach (b) i (d) bardzo wyraźna wiązka została odbita, co można zaobserwować na dolnych wykresach przedstawiających skumulowany promień odbity.
	To co można zauważyć na rys. (d) to zwiększona ilość absorpcji ścieżek w ostatniej warstwie. 
	Wynika to bezpośrednio z zmniejszenia wartości współczynnika rozpraszania. 
	Przy mniejszym rozpraszaniu duża część wiązki pozostaje spójna i dociera do detektora w jednej wiązce, co skutkuje dużym maksimum na detektorze.
	W tym samym przypadku duża część odbitej od interfejsu między drugą a trzecią warstwą powraca do detektora odbicia, przez co widzimy dwa piki na detektorze odbicia.
	Nie obserwujemy natomiast większego, pojedynczego piku na detektorze odbicia w przypadku (b). 
	Może to wynikać z względnie dużej absorpcji, jaka zachodzi w pierwszej warstwie.
	
	\newpage
	
	\subsection*{Wiązka padająca prostopadle}
		
	W drugiej serii obliczeń wiązka światła padała pod kątem prostym na interfejs tzn. $rx~=~0; \; ry~=~1$.
	Zmiany parametrów warstw w kolejnych obliczeniach zostały przedstawione w tabeli \ref{tab:task3}
	
	\begin{table}[H]
		\caption{Zmiany parametrów warstw w kolejnych obliczeniach.}
		\label{tab:task3}
		\centering
		\begin{tabular}{|c|c|c|c|c|c|}
			\hline
			\rowcolor{LightCyan}
			nr. & $n_1$ & $n_2$ & $\mu_{a2}$ & $\mu_{s2}$ & $g_2$ \\ \hline
			5 	& -		& -		& -  & -   & \\ \hline
			6	& 1.0 	& 1.5	& 10 & 210 & - \\ \hline
			7	& 1.0 	& 1.5	& 1  & 210 & - \\ \hline
			8 	& 1.0	& 1.5	& 10 & 210 & 0.75 \\ \hline
		\end{tabular}
	\end{table}
	
	\noindent Wyniki przeprowadzonych obliczeń zostały przedstawione na rys. \ref{fig:task3_calc}
		
	\begin{figure}[H]
		\centering
		\begin{subfigure}{0.49\textwidth}
			\centering
			\includegraphics[width=\textwidth]{../plots/task3_0.pdf}
			\caption{}
		\end{subfigure}
		\begin{subfigure}{0.49\textwidth}
			\centering
			\includegraphics[width=\textwidth]{../plots/task3_1.pdf}
			\caption{}
		\end{subfigure}
		\\
		\begin{subfigure}{0.49\textwidth}
			\centering
			\includegraphics[width=\textwidth]{../plots/task3_2.pdf}
			\caption{}
		\end{subfigure}
		\begin{subfigure}{0.49\textwidth}
			\centering
			\includegraphics[width=\textwidth]{../plots/task3_3.pdf}
			\caption{}
		\end{subfigure}
		\caption{Wyniki propagacji wiązki fotonów dla parametrów podstawowych z zmianami z tabeli \ref{tab:task3}: (a) brak zmian  (b) $n_1$ = 1.0, $n_2$ = 1.5, $\mu_{a2}$ = 10, $\mu_{s2}$ = 210, (c) $n_1$ = 1.0, $n_2$ = 1.5, $\mu_{a2}$ = 1, $\mu_{s2}$ = 210 (d) $n_1$ = 1.0, $n_2$ = 1.5, $\mu_{a2}$ = 10, $\mu_{s2}$ = 210, $g_2$ = 0.75. Na każdym z wykresów przedstawione: u góry - skumulowana transmisja, na dole - skumulowane odbicie, w środku - skumulowana absorpcja podczas propagacji.}
		\label{fig:task3_calc}
	\end{figure}	
	
	\noindent Analizując wiązkę padającą w taki sposób należy zaznaczyć, że może ona niepoprawnie przedstawiać niektóre aspekty zagadnienia propagacji fali.
	Wiązka odbite od różnych warstw interferują z sobą przez co mogą się wzmacniać lub osłabiać, co ma bezpośredni wpływ na skumulowane odbicie dla danej długości fali.
	Należy też zwrócić uwagę na wyrównaną skalę kolorów na każdym z rysunków. 
	Zostało to zrobione w celu lepszego ich porównania, jednak rysunki te znaczącą się różnią w innych skalach kolorów, przez co ciężko je dokładnie porównywać.
	Na pierwszym z rysunków (\ref{fig:task3_calc}(a)) wiązka się rozprasza delikatnie, przez co w zakumulowanej transmisji widoczna jest szeroka krzywa. 
	Na samym środku układu jest pik, świadczący o niecałkowitym rozproszeniu wiązki.
	Duża część wiązki ulega absorpcji już w pierwszej warstwie, na bocznym krańcu. 
	W drugim przypadku (\ref{fig:task3_calc}(b)) zmiana współczynników załamania światła, absorpcji oraz rozpraszania spowodowała rozpraszanie się dużej części wiązki w drugiej warstwie, a właściwie na granicy między drugą, a pierwszą warstwą. 
	Można to zaobserwować poprzez zwiększoną absorpcję w większej części drugiej warstwy. 
	Patrząc na liczbowe wartości skumulowanej transmisji można stwierdzić, że nieznacznie mniej wędrowców dotarło do detektora.
	Część wiązki pozostała ponownie spójna.
	Zmniejszając ponownie współczynnik absorpcji (\ref{fig:task3_calc}(c)) uzyskano wynik podobny do początkowego.
	Wiązka była częściowo spójna, co można zauważyć przez zakumulowaną transmisje. 
	Większa część wiązki odbiła się bezpośrednio od pierwszego interfejsu i wróciła do absorbera. 
	Mniejsza część wiązki uległa absorpcji na ścianach układu.
	W ostatnim z badanych przypadków (\ref{fig:task3_calc}(d)) ponownie widać bardziej rozmytą absorpcję w drugiej warstwie.
	W porównaniu z układem (b) zmieniony został tylko współczynnik anizotropii mający wpływ na kierunek rozproszenia.
	Widoczna jest zmiana rozkładu absorpcji w drugiej warstwie - w przypadku $b$, gdzie współczynnik anizotropii wynosił 0.075 duża część wiązki ulegała absorpcji przy interfejsie z pierwszą warstwą, to znaczy często ulegała rozproszeniu wstecz.
	Gdy zwiększono współczynnik anizotropii większa część wiązki ulega absorpcji przy interfejsie z trzecią warstwą, co świadczy o większym rozproszeniu wprzód.  
	\\
	\\
	W celu weryfikacji działania obliczeń, dla każdego przypadku sprawdzony został warunek 
	\[\frac{T + A + R}{N} = 1,\]
	czyli suma wiązek przetransportowanych, zaabsorbowanych i odbitych musi być równa jeden. 
	Wartości te lekko odbiegały od siebie, jednak były bardzo bliskie jedności:
	\begin{verbatim}
	1 - Suma składowych wynosi: 0.9999999764065
	2 - Suma składowych wynosi: 0.9999999780509999
	3 - Suma składowych wynosi: 0.9999999851275749
	4 - Suma składowych wynosi: 0.9999998685105
	5 - Suma składowych wynosi: 0.9999997396624001
	6 - Suma składowych wynosi: 1.0000002314603
	7 - Suma składowych wynosi: 0.9999998872143999
	8 - Suma składowych wynosi: 1.00000002279705
	\end{verbatim}
	Co potwierdza poprawne wykonanie obliczeń.
	
	\section*{Podsumowanie}
	
	Metoda Monte Carlo w symulowaniu wiązki przechodzącej przez ośrodki mętne jest metodą bardzo wydajną i wydaje się mieć duże zastosowania.
	Poza jej celem, czyli układami medycznymi o trudnej geometrii, nadaje się również do innych układów mętnych.
	Ciekawą modyfikacją tej metody mogłoby być uwzględnienie fazy fotonu oraz jego energii. 
	Zabieg ten pozwoliłby na interferencje fotonów, co pozwoliłoby na badanie układów periodycznych, czy wykorzystujących zjawiska interferencji np. siatki dyfrakcyjne.	
		
		
\end{document}

