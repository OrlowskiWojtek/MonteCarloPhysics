\documentclass[a4paper, 12pt]{article}

\input{/home/linuxwojtek/Desktop/Studia/LaTeX/setup.tex}

\title{\textsc{Monte Carlo: symulacja procesu Wienera, wyznaczanie \\ współczynnika dyfuzji, symulacja procesu dyfuzji i absorpcji} \\ sprawozdanie}
\author{Wojciech Orłowski}
\date{\today}


\begin{document}
	\maketitle
	
	\section*{Wstęp}
	
	Rozważmy dyfuzję układu jednowymiarowego i równanie opisujące taki proces \eqref{diff}
	
	\begin{equation}
		\pdv{u(x,t)}{t} = D \pdv[2]{u(x,t)}{x},
		\label{diff}
	\end{equation}
	gdzie $D$ oznacza współczynnik dyfuzji, a $u(x,t)$ jest funkcją rozkładu lub stężenia w przestrzeni, na przykład rozkładu cząstek w przestrzeni.
	Rozwiązując równanie z narzuconym warunkiem początkowym w postaci \eqref{dirac}
	\begin{equation}
		u(x = x_0,t=0) = \delta(x - x_0), 
		\label{dirac}
	\end{equation}
	gdzie $\delta(x-x_0)$ oznacza dystrybucję delty Diraca, można dojść do rozwiązania postaci \eqref{sol}
	\begin{equation}
		u(x,t) = \frac{1}{\sigma_t \sqrt{2\pi}} \exp(-\frac{(x-x_0)^2}{2\sigma_t^2}),
		\label{sol}
	\end{equation}
	gdzie $\sigma_t = \sqrt{2Dt}$. Rozwiązaniem tego równania jest zatem modulowany czasem rozkład normalny cząstek. Taki sam rozkład uzyskamy, gdy ewolucja cząstki będzie przebiegać z procesem stochastycznym Wienera \eqref{wiener}, to znaczy:
	\begin{equation}
		X_i(t + \delta t) = X_i(t) + \Delta X; \; \; \; \; \Delta X \sim N(0,\delta t)
		\label{wiener}
	\end{equation}
	Czyli każda zmiana położenia cząstki jest losowana za pomocą rozkładu normalnego, określonego funkcją gęstości prawdopodobieństwa \eqref{fgp} (zakładając $\Delta x = x - x_0$)
	\begin{equation}
		f(\Delta x,\Delta t) = \frac{1}{\sigma_{\Delta t}\sqrt{2\pi}} \exp(- \frac{\Delta x^2}{2\sigma_{\Delta t}^2}), \; \; \; \sigma_{\Delta t} = \sqrt{2D\Delta t}
		\label{fgp}
	\end{equation} 
	Znając z kolei rozkład przestrzenny cząstek w czasie można spróbować określić (wcześniej narzucone) współczynniki dyfuzji w każdym z kierunków
	\begin{gather}
		D_{xx} = \frac{\expval{x^2(t)} - \expval{x(t)}^2}{2t}, \\
		D_{yy} = \frac{\expval{y^2(t)} - \expval{y(t)}^2}{2t},
	\end{gather}
	oraz współczynnik $D_{xy}$ będący współczynnikiem określającym korelacje między rozkładami w kierunkach $x$ oraz $y$
	\begin{equation}
		D_{xy} = \frac{\expval{x(t)y(t)} - \expval{x(t)}\expval{y(t)}}{2t}.
	\end{equation}
	Wartości $\expval{x(t)}$, $\expval{y(t)}$, $\expval{x(t)^2}$, $\expval{y(t)^2}$, oraz $\expval{x(t)y(t)}$ to momenty zwykłe, estymowane za pomocą poniższych wzorów
	\begin{gather}
		\expval{x(t)} = \frac{1}{n} \sum_{i=1}^{n} X_i(t) \\ 
		\expval{y(t)} = \frac{1}{n} \sum_{i=1}^{n} Y_i(t) \\
		\expval{x(t)^2} = \frac{1}{n} \sum_{i=1}^{n} X_i^2(t) \\ 
		\expval{y(t)^2} = \frac{1}{n} \sum_{i=1}^{n} Y_i^2(t) \\
		\expval{y(t)x(t)} = \frac{1}{n} \sum_{i=1}^{n} Y_i(t)X_i(t) 
	\end{gather} 
	Wartości współczynnika dyfuzji są zależne od czasu oraz jako proces stochastyczny obarczone pewną niepewnością. Dlatego w celu uzyskania jednej wartości niezależnej od czasu dokonać można uśredniania w pewnym odcinku czasowym $(t_a;t_b)$
	\begin{equation}
		\expval{D_{\alpha \beta}^m} \approx \frac{1}{N_t} \sum_{k=1}^{N_t} D_{\alpha \beta}^m(t_k),
		\label{diff_coefs_mean} 
	\end{equation}
	gdzie $N_t$ to ilość punktów czasowych $t_k$ w odcinku czasowym $(t_a,t_b)$. Aby oszacować odchylenie standardowe korzystamy z wzoru \eqref{std_dev}
	\begin{equation}
		\sigma_{\expval{D_{\alpha \beta}}} = \sqrt{\frac{\expval{D_{\alpha \beta}^2} - \expval{D_{\alpha \beta}}^2}{N_t}}. 
		\label{std_dev}
	\end{equation}
	
	\noaka \krecha
	
	\noaka Drugim badanym problemem był proces dyfuzji w układzie zamkniętym. 
	Do takiego układu wprowadzane były cząstki z źródła o wydajności
	\begin{equation}
		\omega = \frac{\Delta n}{\Delta t}.
	\end{equation}
	W takim układzie $\Delta n$ właściwie oznacza maksymalną liczbę cząstek, jaką może wydać źródło w czasie $\Delta t$.
	Następnie cząstki ulegają dyfuzji z schematem Wienera, z tą różnicą, że są one zamknięte w układzie o kształcie koła.
	Cząstka odbija się od ściany zgodnie z prawem 
	odbicia.
	Jeżeli trajektoria cząstki natrafi na absorber (o promieniu $R_a$) zostaje pochłonięta. 
	Odbicie cząstki oraz jej absorpcja zostały zaimplementowane w programie za pomocą funkcji dostarczonej przez prowadzącego zajęcia:
	\begin{verbatim}
		void particle_translation(double & x1,double & y1, double & x2,double & y2, 
		double xr,double yr, double Rr, double xa,double ya, double Ra, 
		int & exist ,double & length)
	\end{verbatim}
	
	\newpage
	
	\section*{Wyniki}
	
	\subsection*{Układ otwarty (proces Wienera)}
	
	W zadanym problemie należało zbadać dyfuzje cząstek w układzie otwartym, a następnie obliczyć odpowiednie współczynniki dyfuzji dla różnych ilości cząstek w układzie otwartym.
	Dyfuzja cząstek była badana dla różnej ich ilości. Położenie początkowe cząstek zostało ustawione na (0,0).
	Położenia cząstek w kilku czasach symulacji dla każdej z liczby cząstek zostały przedstawione na rys. \ref{1_loc}
	
	\begin{figure}[H]
		\centering
		\begin{subfigure}{0.49\textwidth}
			\centering
			\includegraphics[width=\textwidth]{../plots/diff_times_N=100.png}
			\caption{}
		\end{subfigure}
		\begin{subfigure}{0.49\textwidth}
			\centering
			\includegraphics[width=\textwidth]{../plots/diff_times_N=1000.png}
			\caption{}
		\end{subfigure}
		\\
		\begin{subfigure}{0.49\textwidth}
			\centering
			\includegraphics[width=\textwidth]{../plots/diff_times_N=10000.png}
			\caption{}
		\end{subfigure}
		\begin{subfigure}{0.49\textwidth}
			\centering
			\includegraphics[width=\textwidth]{../plots/diff_times_N=100000.png}
			\caption{}
		\end{subfigure}
		\caption{Położenia cząstek w 1.0 s, 5.0 s oraz 20.0 s od początku symulacji dla (a) 100 cząstek (b) 1000 cząstek (c) 10000 cząstek (d) 100000 cząstek.}
		\label{1_loc}
	\end{figure}
	
	\noaka Dla większej ilości cząstek, można zauważyć wysokie podobieństwo między wykonywaną symulacją, a modulowanym rozkładem normalnym. 
	Dla 100 cząstek nie można wskazać konkretnych wniosków, niektóre cząstki zdążyły bardzo szybko się przemieścić, nawet w czasie jednej sekundy.
	Warto w takich symulacjach stosować większą liczbę próbek, w celu zwiększenia pewności danych statystycznych.
	Obliczone współczynniki dyfuzji w każdej chwili czasowej zostały przedstawione na rys. \ref{diff_coefs}.
	\begin{figure}[H]
	\centering
	\begin{subfigure}{0.49\textwidth}
		\centering
		\includegraphics[width=\textwidth]{../plots/D_N=100.png}
		\caption{}
	\end{subfigure}
	\begin{subfigure}{0.49\textwidth}
		\centering
		\includegraphics[width=\textwidth]{../plots/D_N=1000.png}
		\caption{}
	\end{subfigure}
	\\
	\begin{subfigure}{0.49\textwidth}
		\centering
		\includegraphics[width=\textwidth]{../plots/D_N=10000.png}
		\caption{}
	\end{subfigure}
	\begin{subfigure}{0.49\textwidth}
		\centering
		\includegraphics[width=\textwidth]{../plots/D_N=100000.png}
		\caption{}
	\end{subfigure}
	\caption{Obliczone współczynniki dyfuzji $D_{xx}$, $D_{yy}$ oraz $D_{xy}$ w różnych chwilach czasowych dla (a) 100 cząstek (b) 1000 cząstek (c) 10000 cząstek (d) 100000 cząstek.}
	\label{diff_coefs}
\end{figure}

\noaka Dla małej liczby cząstek można zauważyć duże fluktuacje między wartościami właściwymi (1 oraz 0).
Fluktuacje tego typu, przypominające szum są typowe dla obliczeń stochastycznych. 
Dla większej ilości próbek są one dużo gładsze.
Dla 100000 cząstek współczynniki dyfuzji $D_{xx}$ oraz $D_{yy}$ szybko osiągają wartość dokładną, zadaną w programie oraz równą jeden, a współczynnik korelacji $D_{xy}$ szybko wskazuje na brak korelacji między kierunkami $x$ i $y$.
\\
\\
Dla całego ciągu danych współczynników dyfuzji można obliczyć konkretne wartości uśredniając wartość współczynnika po całym zakresie oraz szacując niepewność zgodnie z wzorami \eqref{diff_coefs_mean} oraz \eqref{std_dev}.
Przybliżone wartości dla każdej ilości cząstek wyniosły:
\begin{verbatim}
			N = 100
			====Diffusion coefficients values===
			D\_xx: 0.89317 std. err.: 0.00294913
			D\_yy: 0.911995 std. err.: 0.00379536
			D\_xy: -0.0712799 std. err.: 0.0015523
			====================================
			
			N = 1000
			====Diffusion coefficients values===
			D\_xx: 1.00779 std. err.: 0.00156458
			D\_yy: 0.950738 std. err.: 0.000616031
			D\_xy: 0.0405798 std. err.: 0.000542106
			====================================
			
			N = 10000
			====Diffusion coefficients values===
			D\_xx: 0.999638 std. err.: 0.000374029
			D\_yy: 1.00111 std. err.: 0.000258967
			D\_xy: 0.00490585 std. err.: 0.000214109
			====================================
			
			N = 100000
			====Diffusion coefficients values===
			D\_xx: 0.998716 std. err.: 7.57354e-05
			D\_yy: 0.999415 std. err.: 9.74718e-05
			D\_xy: -0.00327316 std. err.: 8.63936e-05
			====================================
\end{verbatim} 
Zgodnie z oczekiwaniami wartości odchyleń maleją wraz z wzrostem ilości symulowanych cząstek.  Odchylenie standardowe średniej powinno maleć proporcjonalnie do pierwiastka z liczby próbek. Podobną zależność można tutaj zaobserwować. Dokonując aproksymacji algorytmem Levenberga-Marquadta można się przekonać, że uzyskane odchylenia standardowe spełniają tą zależność (rys. \ref{fit})
	
\begin{figure}[h]
	\begin{subfigure}{0.49\textwidth}
		\centering
		\includegraphics[width=\textwidth]{../jl_analasysis/std_dxx.png}
		\caption{}
	\end{subfigure}
	\begin{subfigure}{0.49\textwidth}
		\centering
		\includegraphics[width=\textwidth]{../jl_analasysis/std_dyy.png}
		\caption{}
	\end{subfigure}
	\caption{Aproksymowane krzywe wraz z naniesionymi odchyleniami wynikającymi z obliczeń.}
	\label{fit}
\end{figure}

	
\newpage
	
\subsection*{Układ zamknięty - źródło i absorbent}
	
W badanych przypadkach ustalone zostały różne parametry źródła ($\omega \in \{10; 50; 100\}$)i absorbera ($R_a \in \{0.1;0.5\}$). Rozkład cząstek w kilku chwilach symulacji w zależności od zadanych parametrów został przedstawiony na rys. \ref{rozklad_ra1}

\begin{figure}[H]
	\centering
	\begin{subfigure}{0.30\textwidth}
		\centering
		\includegraphics[width=\textwidth]{../plots/task_2/it=100om=10ra=0.100000.png}
		\caption{}
	\end{subfigure}
	\begin{subfigure}{0.30\textwidth}
		\centering
		\includegraphics[width=\textwidth]{../plots/task_2/it=1000om=10ra=0.100000.png}
		\caption{}
	\end{subfigure}
	\begin{subfigure}{0.30\textwidth}
		\centering
		\includegraphics[width=\textwidth]{../plots/task_2/it=5000om=10ra=0.100000.png}
		\caption{}
	\end{subfigure}
	\\
	\begin{subfigure}{0.30\textwidth}
		\centering
		\includegraphics[width=\textwidth]{../plots/task_2/it=100om=50ra=0.100000.png}
		\caption{}
	\end{subfigure}
	\begin{subfigure}{0.30\textwidth}
		\centering
		\includegraphics[width=\textwidth]{../plots/task_2/it=1000om=50ra=0.100000.png}
		\caption{}
	\end{subfigure}
	\begin{subfigure}{0.30\textwidth}
		\centering
		\includegraphics[width=\textwidth]{../plots/task_2/it=5000om=50ra=0.100000.png}
		\caption{}
	\end{subfigure}
	\\
	\begin{subfigure}{0.30\textwidth}
		\centering
		\includegraphics[width=\textwidth]{../plots/task_2/it=100om=100ra=0.100000.png}
		\caption{}
	\end{subfigure}
	\begin{subfigure}{0.30\textwidth}
		\centering
		\includegraphics[width=\textwidth]{../plots/task_2/it=1000om=100ra=0.100000.png}
		\caption{}
	\end{subfigure}
	\begin{subfigure}{0.30\textwidth}
		\centering
		\includegraphics[width=\textwidth]{../plots/task_2/it=5000om=100ra=0.100000.png}
		\caption{}
	\end{subfigure}
	
	\caption{Rozkład cząstek w czasie $t=10$ (a),(d),(g); $t=100$ (b),(e),(h); $t=500$ (c),(f),(i) dla $R_a = 0.1$ oraz źródła $\omega = 10.0$ (a),(b),(c); $\omega = 50.0$ (d),(e),(f) oraz $\omega = 100$ (g),(h),(i). Niebieskim kolorem zaznaczone źródło, czerwonym absorber.}
	\label{rozklad_ra1}
\end{figure}

\noaka Zwiększenie wydajności źródła znacznie zwiększa ilość cząstek w danej chwili czasowej. Widać też, że położenie absorbera wpływa mało znacząco na rozkład przestrzenny cząstek, jednak jest ich mniej po stronie absorbera. Może to wynikać z położenia źródła po lewej stronie układu. Podobny rozkład przestrzenny jednak dla większego absorbera został przedstawiony na rys. \ref{rozklad_ra2}

\begin{figure}[H]
	\centering
	\begin{subfigure}{0.30\textwidth}
		\centering
		\includegraphics[width=\textwidth]{../plots/task_2/it=100om=10ra=0.500000.png}
		\caption{}
	\end{subfigure}
	\begin{subfigure}{0.30\textwidth}
		\centering
		\includegraphics[width=\textwidth]{../plots/task_2/it=1000om=10ra=0.500000.png}
		\caption{}
	\end{subfigure}
	\begin{subfigure}{0.30\textwidth}
		\centering
		\includegraphics[width=\textwidth]{../plots/task_2/it=5000om=10ra=0.500000.png}
		\caption{}
	\end{subfigure}
	\\
	\begin{subfigure}{0.30\textwidth}
		\centering
		\includegraphics[width=\textwidth]{../plots/task_2/it=100om=50ra=0.500000.png}
		\caption{}
	\end{subfigure}
	\begin{subfigure}{0.30\textwidth}
		\centering
		\includegraphics[width=\textwidth]{../plots/task_2/it=1000om=50ra=0.500000.png}
		\caption{}
	\end{subfigure}
	\begin{subfigure}{0.30\textwidth}
		\centering
		\includegraphics[width=\textwidth]{../plots/task_2/it=5000om=50ra=0.500000.png}
		\caption{}
	\end{subfigure}
	\\
	\begin{subfigure}{0.30\textwidth}
		\centering
		\includegraphics[width=\textwidth]{../plots/task_2/it=100om=100ra=0.500000.png}
		\caption{}
	\end{subfigure}
	\begin{subfigure}{0.30\textwidth}
		\centering
		\includegraphics[width=\textwidth]{../plots/task_2/it=1000om=100ra=0.500000.png}
		\caption{}
	\end{subfigure}
	\begin{subfigure}{0.30\textwidth}
		\centering
		\includegraphics[width=\textwidth]{../plots/task_2/it=5000om=100ra=0.500000.png}
		\caption{}
	\end{subfigure}
	
	\caption{Rozkład cząstek w czasie $t=10$ (a),(d),(g); $t=100$ (b),(e),(h); $t=500$ (c),(f),(i) dla $R_a = 0.5$ oraz źródła $\omega = 10.0$ (a),(b),(c); $\omega = 50.0$ (d),(e),(f) oraz $\omega = 100$ (g),(h),(i). Niebieskim kolorem zaznaczone źródło, czerwonym absorber.}
	\label{rozklad_ra2}
\end{figure}
 
 \noaka Po zwiększeniu promienia absorbera można zauważyć spadek ilości cząstek w prawej stronie układu. Zagęszczenie cząstek w pobliżu źródła zmalało, jednak nie tak znacząco. Wykresy zależności całkowitej ilości cząstek od iteracji zostały przedstawione na rys. \ref{n_od_it}
 
\begin{figure}[H]
	\centering
	\begin{subfigure}{0.4\textwidth}
		\centering
		\includegraphics[width=\textwidth]{../plots/task_2/n_for_om=10ra=0.100000.png}
		\caption{}
	\end{subfigure}
	\begin{subfigure}{0.4\textwidth}
		\centering
		\includegraphics[width=\textwidth]{../plots/task_2/n_for_om=50ra=0.100000.png}
		\caption{}
	\end{subfigure}
	\\
	\begin{subfigure}{0.4\textwidth}
		\centering
		\includegraphics[width=\textwidth]{../plots/task_2/n_for_om=100ra=0.100000.png}
		\caption{}
	\end{subfigure}
	\begin{subfigure}{0.4\textwidth}
		\centering
		\includegraphics[width=\textwidth]{../plots/task_2/n_for_om=10ra=0.500000.png}
		\caption{}
	\end{subfigure}
	\\
	\begin{subfigure}{0.4\textwidth}
		\centering
		\includegraphics[width=\textwidth]{../plots/task_2/n_for_om=50ra=0.500000.png}
		\caption{}
	\end{subfigure}
	\begin{subfigure}{0.4\textwidth}
		\centering
		\includegraphics[width=\textwidth]{../plots/task_2/n_for_om=100ra=0.500000.png}
		\caption{}
	\end{subfigure}
	\caption{Całkowita ilość cząstek w układzie w zależności od iteracji dla $R_a = 0.1$ (a),(b),(c); $R_a = 0.5$ (d),(e),(f); $\omega = 10$ (a),(d); $\omega = 50$ (b),(e); $\omega = 100$ (c),(f).}
	\label{n_od_it}
\end{figure}

\noaka Po pewnym czasie dla każdego z przypadków ilość cząstek w układzie ulega stabilizacji i fluktuuje wokół pewnej wartości.
Dla największej wydajności źródła i najmniejszego promienia absorbera po pewnym czasie w układzie jest maksymalna ilość cząstek. 
Stan równowagi jaki jest osiągany w układzie zależy bezpośrednio od jego parametrów - im większa wydajność źródła tym więcej cząstek, a czym większy promień to mniej cząstek po stabilizacji.
Wydaje się, że istnieje liniowa zależność między wydajnością źródła a ilością cząstek w układzie. 
Można to zauważyć na przykładzie układu z większym absorberem (tabela \ref{tab1}).
\begin{table}[h]
	\caption{Ilość cząstek w układzie w zależności od wydajności źródła dla większego promienia absorbera.}
	\label{tab1}
	\centering
	\begin{tabular}{|c|c|}
		\hline
		$\omega$ & $n$ po stabilizacji \\ \hline
		10 & 500 \\ \hline
		50 & 2500 \\ \hline
		100 & 5000 \\ \hline
	\end{tabular}
\end{table}
Można zatem postulować, że w przypadku największej wydajności źródła dla $R_a = 0.1$ ilość cząstek w układzie po stabilizacji powinna wynosić około 12000 cząstek (co jest większe od maksymalnej dozwolonej ilości cząstek).
\\
\\
Ciekawym faktem też jest zależność występujących fluktuacji w zależności od $\omega$.
Dla większej wydajności źródła w układzie jest więcej cząstek, które ciężej ulegają fluktuacją i dostarczają lepszą informacje statystyczną. 
\\
\\
Zwiększając promień absorbera pięciokrotnie zmniejszamy ilość cząstek w układzie o ponad połowę.
Dla najmniejszych wydajności źródła ilość cząstek zmalała z około 1200 do około 500. 
Ilość cząstek w układzie powinna być bezpośrednio zależna od powierzchni absorbera, która w tym przypadku wzrosła 25-krotnie. 
Natomiast ilość cząstek zmalała około 2.4-krotnie. 
Aby znaleźć dokładniejszą zależność między powierzchnią absorbera a ilością cząstek w układzie po stabilizacji należałoby wykonać większą liczbę eksperymentów Monte Carlo.

\section*{Podsumowanie}

Za pomocą prostego procesu stochastycznego (procesu Wienera) jesteśmy w stanie dokonywać symulacji dyfuzji w układzie.
W wykonywanych obliczeniach nie brano pod uwagę oddziaływań między cząsteczkami, jednak udało się uzyskać ciekawe wyniki, zarówno w procesie otwartym jak i zamkniętym.
Obliczenia te jednak są ściśle zależne od ilości cząstek w symulacji, co można zauważyć po obliczonych odchyleniach standardowych.
Zwiększenie ilości symulowanych obiektów znacząco wydłuża czas symulacji, nawet nie biorąc pod uwagę oddziaływań międzycząsteczkowych.

	
\end{document}

