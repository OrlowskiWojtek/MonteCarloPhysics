\documentclass[a4paper, 12pt]{article}

\input{~/Desktop/Studia/LaTeX/setup.tex}
\usepackage{amsmath}


\author{Wojciech Orłowski}
\date{\today}
\title{\textsc{Monte Carlo: propagacja sygnału elektrycznego w linii transmisyjnej, problem zależny od czasu}\\ - sprawozdanie}


\begin{document}
	\maketitle
	
	\section*{Wstęp}
	
	Jednym z podstawowych zagadnień elektroniki jest opis propagacji sygnału. 
	W linii transmisyjnej (np. linii długiej) propagację sygnału opisuje układ równań różniczkowych:
	\begin{align}
		\pdv{u}{x}	&= -L\pdv{i}{t} - Ri, \\
		\pdv{i}{t}  &= -C\pdv{u}{t} - Gu.
	\end{align}	
	Układ ten opisuje zmiany napięcia ($u$) oraz prądu ($i$) w czasie i położeniu. Współczynniki $L,R,C,G$ to kolejno indukcyjność, opór, pojemność i konduktancja.
	Te dwa równania różniczkowe są od siebie zależne i w celu ich rozwiązania dokonuje się transformacji
	\begin{align}
		f(x,t) &= \frac{1}{2} (u + R_0 i), \\
		b(x,t) &= \frac{1}{2} (u - R_0 i) ,
	\end{align}
	gdzie $R_0 = \sqrt{\frac{L}{C}}$. Ważnym faktem jest zależność $f+b = u$, pozwalająca nam w końcowym etapie uzyskać z powrotem informację o napięciu. Znając zatem $f$ oraz $b$ znamy $u$ oraz $i$, zatem można rozwiązać alternatywny układ różniczkowy
	\begin{align}
		\pdv{f}{t} &= -c\pdv{f}{x} + \lambda b - (\lambda + \mu)f, \\
		\pdv{b}{t} &= c\pdv{b}{x} + \lambda f - (\lambda + \mu)b, 
	\end{align}
	gdzie stałe są wyrażone wzorami $c = \frac{1}{\sqrt{LC}},\;\mu = \frac{G}{C},\;\lambda = \frac{1}{2}\left(\frac{R}{L} - \frac{G}{C}\right) \mu.$
	Powyższy układ możemy zapisać w postaci układu równań całkowych:
	\begin{align}
		f(x,t) &= f_0(x,t) \exp(-(\lambda + \mu)t)+\frac{\lambda}{\lambda + \mu} \int_{0}^{t} (\lambda + \mu)e^{-(\lambda+\mu)s} b(x-cs,t-s)\dd s, \\
		b(x,t) &= b_0(x,t) \exp(-(\lambda + \mu)t)+\frac{\lambda}{\lambda + \mu} \int_{0}^{t} (\lambda + \mu)e^{-(\lambda+\mu)s} f(x+cs,t-s)\dd s.
	\end{align}
	Na podstawie postaci funkcji $f$ i $b$ oraz ich występowania można je interpretować jako rozwiązanie poruszające się w prawą ($f$) i lewą ($b$) stronę.
	Całki te naturalnie wyglądają jak pewien rozkład wartości z odpowiednią funkcją gęstości prawdopodobieństwa równą
	\begin{equation}
		p(s) = (\lambda + \mu)e^{-(\lambda+\mu)s},
	\end{equation}	
	która ma postać już znormalizowaną.
	Znając odpowiednie prawdopodobieństwo można wywoływać funkcje $f$ i $b$ z określonym prawdopodobieństwem rekurencyjnie.
	W układzie zostały nałożone warunki brzegowe (na początek układu - punkt $0$ i na koniec układu - punkt $l$):
	\begin{itemize}
		\item źródło napięcia o oporze własnym $R_g$,
		\item opornik na końcu o oporze $R_l$.
	\end{itemize}
	Matematycznie generator nakłada na napięcie w początku układu pewną wartość
	\begin{equation}
		u(0,t) = V_g(t) - R_g \cdot i(0,t),
	\end{equation}
	gdzie $V_g(t)$ to funkcja opisująca przebieg czasowy sygnału. Na wyjściu działanie opornika zapisujemy jako
	\begin{equation}
		i(l,t) = \int_{-\infty}^{\infty} u(l,\tau) h(t-\tau) \dd \tau,	
	\end{equation}
	gdzie $h(t-\tau)$ jest odpowiedzią układu na impuls o wartości $\delta(t-\tau)/R_l$. Po wcześniej dokonanej transformacji otrzymano warunki
	\begin{align}
		f(0,t) &= \zeta V_g(t) + \Gamma_g b(0,t), \\
		b(l,t) &= \Gamma_l f(l,t),
	\end{align}
	gdzie współczynniki są wyrażone wzorami
	\begin{gather*}
		R_0 = \sqrt{\frac{L}{C}}, \\
		\zeta = \frac{R_0}{R_0 + R_g}, \\
		\Gamma_g = \frac{R_g - R_0}{R_g + R_0}, \\
		\Gamma_l = \frac{R_l - R_0}{R_l + R_0}.
	\end{gather*}
	Jeżeli rozwiązanie dojdzie do początku układ (np. po wcześniejszym odbiciu od końca) to następuje odbicie sygnału z odpowiednim współczynnikiem i dodatkiem generowanego sygnału. Na prawym brzegu również występuje odpowiednie odbicie.
	\\
	\\
	W algorytmie Monte Carlo nadpisujemy wartości całek naprzemiennie zgodnie z klasycznym całkowaniem Monte Carlo. Losowana jest zmienna $s$ wraz z rozkładem eksponecjalnym, na podstawie której zmieniany jest czas i położenie.   
	
	\section*{Wyniki}
	
	Podczas ćwiczenia wykonano obliczano napięcia w linii transmisyjnej metodą Monte Carlo w różnych chwilach czasowych. 
	Wybrane parametry ćwiczenia zostały przedstawione w tabeli \ref{tab1}.
	\begin{table}[H]
		\centering
		\caption{Parametry linii ustawione w ćwiczeniu.}
		\label{tab1}
		\begin{tabular}{|c|c|c|c|c|c|c|}
			\hline
			$L$ [\textmu H] &  $C$ [pF] & $R$ [\textohm] & $G$ [mS] & $l$ [m] & $R_l$ [\textohm] & $R_g$ [\textohm] \\  
			\hline
			0.25 & 100 & 12.5 & 0.5 & 2 & 12.5 & 75.0 \\
			\hline
		\end{tabular}
	\end{table}
	
	\noaka Sygnał generowany na brzegu miał postać
	\begin{equation}
		V_g(t) = \sin(2\pi \nu t) \exp(-\frac{(t-t_0)^2}{2\sigma^2}),
	\end{equation}
	gdzie parametry sygnału mają wartości przedstawione w tabeli \ref{tab2}.
	\begin{table}[H]
		\centering
		\caption{Parametry sygnału ustawione w ćwiczeniu}
		\label{tab2}
		\begin{tabular}{|c|c|c|}
			\hline
			$\nu$ [GHz]  &  $t_0$ [ns] & $\sigma$ [ns] \\ \hline
			1 & 7.5 & 0.75 \\
			\hline
		\end{tabular}
	\end{table}
	
	\noindent Obliczenia wykonywano dla czasów $t = 10,15,25,35,50$ ns oraz $npaths = 10^3,10^4,10^5$.
	Wyniki dla czasu $t = 10$ ns zostały przedstawione na rys. \ref{fig:10}.
	\begin{figure}[H]
		\centering
		\includegraphics[width=\textwidth]{../plots/whole_plots/time_10.png}
		\caption{Wyniki obliczeń wraz z błędami dla różnej ilości ścieżek dla czasu $t = 10$ ns.}
		\label{fig:10}
	\end{figure}
	
	\noindent Obliczenia metodą Monte Carlo są zgodne z dużą dokładnością z wynikiem dokładnym. 
	Największe błędy można zauważyć przy sygnale właściwym, to znaczy przy dużym napięciu.
	Nawet ten błąd można wyraźnie zredukować zwiększając ilość ścieżek. 
	W dalszych krokach czasowych sygnał będzie się odbijał od ścianek i jednocześnie zmniejszał swoją amplitudę. 
	Dla kolejnych chwil czasowych wartości zostały przedstawione na rysunkach \ref{fig:15}, \ref{fig:25}, \ref{fig:35}, \ref{fig:50}.
	\begin{figure}[H]
		\centering
		\includegraphics[width=\textwidth]{../plots/whole_plots/time_15.png}
		\caption{Wyniki obliczeń wraz z błędami dla różnej ilości ścieżek dla czasu $t = 15$ ns.}
		\label{fig:15}
	\end{figure}
	
	\noindent Po pięciu nanosekundach amplituda sygnału się nie zmniejszyła. 
	Kształt sygnału również się nie zmienił.
	Różnice są również tego samego rzędu wielkości.
	Wysokie różnice są tam, gdzie jest duża amplituda fali (duże napięcie).
	Maleją one w podobnym tempie jak wcześniej.
	Jednak mimo podobnego rzędu wielkości, są one tym razem bardziej widoczne dla najmniejszej ilości ścieżek.
	

	\begin{figure}[H]
		\centering
		\includegraphics[width=\textwidth]{../plots/whole_plots/time_25.png}
		\caption{Wyniki obliczeń wraz z błędami dla różnej ilości ścieżek dla czasu $t = 25$ ns.}
		\label{fig:25}
	\end{figure}
	
	\noaka Po kolejnych dziesięciu sekundach obserwujemy fale już odbitą od prawego końca układu.
	Amplituda fali zmalała, a jej kształt został zachowany. 
	Błędy są cały czas małe, zauważalne gołym okiem tylko dla najmniejszej ilości ścieżek.
	
	\begin{figure}[H]
		\centering
		\includegraphics[width=\textwidth]{../plots/whole_plots/time_35.png}
		\caption{Wyniki obliczeń wraz z błędami dla różnej ilości ścieżek dla czasu $t = 35$ ns.}
		\label{fig:35}
	\end{figure}
	
	\noindent Gdy fala odbiła się ponownie od początku układu jej amplituda znacznie zmalała.
	Mimo, że wartości bezwzględne błędu bardzo się nie zmieniły, to są one bardzo widoczne dla najmniejszej ilości ścieżek oraz lekko widoczne dla 10000 ścieżek.
	
	\begin{figure}[H]
		\centering
		\includegraphics[width=\textwidth]{../plots/whole_plots/time_50.png}
		\caption{Wyniki obliczeń wraz z błędami dla różnej ilości ścieżek dla czasu $t = 50$ ns.}
		\label{fig:50}
	\end{figure}
	
	\noindent Dla czasu 50 ns obserwujemy falę już po wielu odbiciach, której amplituda zmalała już wielokrotnie. 
	Błędy bezwzględne zmalały znacząco, jednak porównując wyniki są one bardziej zaszumione niż wcześniej.
	Zatem błąd względny wzrósł. 
	Jesteśmy w stanie zauważyć niedokładności dla obliczeń z największą ilością ścieżek.
	\\
	\\
	Poza zwiększającym się błędem względnym wraz z czasem trwania symulacji obliczenia też trwają coraz dłużej.
	Można łatwo sprawdzić, że czas obliczeń rośnie liniowo wraz z czasem trwania symulacji.
	Dla całego zestawu obliczeń (tj. dla $10^3$, $10^4$ oraz $10^5$ łańcuchów - czyli właściwie dla 111000 łańcuchów) czasy obliczeń w zależności od czasu propagacji fali zostały przedstawione na rys. \ref{fig:times}.
	
	\begin{figure}[H]
		\centering
		\includesvg[width=0.6\textwidth]{../plots/times.svg}
		\caption{Czas obliczeń wraz z dopasowaną prostą.}
		\label{fig:times}
	\end{figure}
	
	\noaka Co się zgadza z teorią analityczną zagadnienia, według której koszt obliczeniowy jest wprost proporcjonalny do $t$:
	\begin{equation}
		C \approx N\frac{t}{\expval{s}},
	\end{equation}
	gdzie $N$ to liczba wygenerowanych ścieżek, a $\expval{s} = \frac{1}{\lambda + \mu}$. 
	
	\section*{Podsumowanie}
	
	Algorytm stochastyczny rozwiązania tego problemu stanowi świetną alternatywę dla metod różnic skończonych.
	Pozwala on znaleźć w wartość napięcia w każdym miejscu i czasie, niezależnie od innych obliczeń.
	Oznacza to, że aby poznać wartość napięcia w punkcie $x_0$ nie potrzeba znać wartości napięcia w żadnych innych punktach.
	W metodach siatkowych zawsze rozwiązuje się problem dla całego zestawu wartości.
	Ponadto rozwiązanie otrzymujemy z dużą dokładnością, którą w prostu sposób można zwiększyć.
	Algorytm ma też potencjał do łatwego zrównoleglenia obliczeń.
	
	
\end{document}

