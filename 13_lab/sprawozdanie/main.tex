\documentclass[a4paper, 12pt]{article}

\input{~/Desktop/Studia/LaTeX/setup.tex}
\usepackage{amsmath}


\author{Wojciech Orłowski}
\date{\today}
\title{\textsc{Monte Carlo: symulacja fali termicznej, problem Riemanna}\\ - sprawozdanie}

\begin{document}
	\maketitle
	
	\section*{Wstęp}
	
	W opisywanym problemie opisujemy fale termobaryczną za pomocą symulacji bezpośredniej gazu metodą Monte Carlo.
	W programie skorzystano w wcześniejszego programu, użytego podczas laboratorium nr 12. 
	Tym razem klasę \texttt{DSMC\_2D} wykorzystano w celu monitorowania propagacji fali cząstek w układzie.
	Większą ilość cząstek o większej temperaturze umieszczono z lewej strony układu, a mniejsza ilość cząstek, mająca stanowić atmosferę, została umieszczona z prawej strony.
	W momencie rozpoczęcia symulacji fala większej gęstości cząstek przemierza z lewej strony w prawo.
	Fala po spotkaniu z prawym końcem obszaru obliczeniowego odbija się i propaguje dalej w drugą stronę.
	
	\section*{Wyniki}
	
	Na początku wygenerowano atomy w lewej i prawej części układu.
	W tym celu zainicjalizowano obiekt klasy \texttt{DSMC\_2D} i wczytano odpowiednie pliki wejściowe.
	Z lewej strony znalazło się $8 \cdot 10^5$ atomów o prędkości zgodnej z rozkładem Maxwella 2D dla temperatury 10000, a z prawej strony umiejscowiono $1 \cdot 10^5$ atomów o prędkości mniejszej, równej 300.
	Jednostki w zadaniu zostały pominięte, ze względu na ustawienie stałej Boltzmanna na wartość równą $1$ - stworzyło to nowy układ jednostek arbitralnych. 
	Takie ustawienie jednostek pozwoliło na zaobserwowanie fali.
	Taka pojedyncza fala dała dużą liczbę wyników do analizy. 
	\\
	\\
	Wszystkie brzegi stanowiły warunek brzegowy Neumanna - cząstki się od nich odbijały.
	Promień cząsteczek ustawiono na $10^{-4}$ - jest to też zawyżona wartość promienia efektywnego, mający być przybliżeniem wielu cząsteczek.
	Jeden atom w symulacji stanowi odzwierciedlenie dla wielu rzeczywistych atomów.
	Jest to jedyny sposób aby odzwierciedlić oddziaływania rzeczywistej atmosfery, w której ilość atomów na jednostkę objętości jest rzędu stałej Avogadra (6.02 $\cdot 10^{23}$).
	Obliczenia dla takiej ilości cząsteczek są niemożliwe do zrealizowania ze względu na aktualnie istniejącą moc obliczeniową.
	Masa efektywna cząsteczek została ustawiona na 1.
	\\
	\\
	Najpierw zostanie przeanalizowany w sposób jakościowy wygląd układu czyli rozkład cząstek w przestrzeni (rys. \ref{loc_1} oraz \ref{loc_2}).

	\begin{figure}[H]
		\centering
		\includegraphics[width=0.9\textwidth]{../plots/pos.pdf}
		\caption{Rozkład cząstek przed zderzeniem z prawym brzegiem.}
		\label{loc_1}
	\end{figure}
	
	\begin{figure}[H]
		\centering
		\includegraphics[width=0.9\textwidth]{../plots/pos_end.pdf}
		\caption{Rozkład cząstek po zderzeniu z prawym brzegiem}
		\label{loc_2}
	\end{figure}
	
	\noindent Na ilustracjach przedstawiających rozkład cząstek został przedstawiony co 20 atom. 
	Taki zabieg umożliwia lepsze zaobserwowanie gęstości cząstek w danym miejscu.
	Ponadto wygenerowanie wysokiej jakości ilustracji wektorowej z taką ilością danych spowodowałoby powstanie pliku o bardzo dużej zajmowanej pamięci.
	\\
	\\
	Na podstawie tych rysunków można stwierdzić, żę fala odbiła się od prawego brzegu w okolicy 500 - 600 iteracji, jednak nie było to zjawisko natychmiastowe, a trwające cały okres czasu.
	Duża ilość masy poruszała się w prawą stronę, a następnie przez pewien okres czasu odbijała.
	Po pierwszym odbiciu od prawej ścianki większa ilość cząstek znajdowała się po prawej stronie.
	Takie zachowanie częściowo przypomina tłumiony oscylator.
	Wiadomo, że w przypadku układu o warunkach brzegowych Neumanna ostatecznie ulegnie on stabilizacji.
	\\
	\\
	Dużo lepszej analizy można dokonać za pomocą obliczonych wartości ciśnienia, temperatury, gęstości i prędkości cząsteczek.
	Uśrednione rozkłady tych wielkości na zmiennej przestrzennej $x$ zostały przedstawione na rys. \ref{nptv_1} (początek symulacji), \ref{nptv_2} (zderzenie się z brzegiem) oraz \ref{nptv_3} (fala odbita).
	
	\begin{figure}[H]
		\centering
		\includegraphics[width=\textwidth]{../plots/nptv_301.pdf}
		\caption{Rozkłady wielkości opisujących układ na początku symulacji.}
		\label{nptv_1}
	\end{figure}

	\noindent W pierwszej iteracji widoczny jest równy podział na dwa obszary.
	W lewym i prawym obszarze są stałe gęstość, temperatura oraz ciśnienie. 
	Uśredniona prędkość jest zerowa w obu obszarach (cząstki poruszają się we wszystkie strony), poza samym środkiem obszaru symulacyjnego. 
	Tam prędkość jest delikatnie dodatnia, co oznacza, że cząstki z bardziej zagęszczonego obszaru poruszają się w stronę mniej zagęszczonego, gdyż po prostu tam jest miejsce. 
	W pierwszych chwilach trwania symulacji gęstość cząstek się wyrównuje na obszarze. 
	Podobnie zachowuje się ciśnienie. 
	Temperatura natomiast jest większa na czele fali (tam gdzie porusza się gęstość cząsteczek). 
	Prędkość atomów jest równomiernie rozłożona na pewnym obszarze stanowiącym obszar od środka układu do czoła fali temperatury. 
	Odpowiadać to może sytuacji, w której cząsteczki poruszając się w prawo robią \textit{miejsce}, aby cząsteczki bardziej lewej strony układu mogły się poruszyć w  prawą stronę. 

	\begin{figure}[H]
	\centering
		\includegraphics[width=\textwidth]{../plots/nptv_701.pdf}
		\caption{Rozkłady wielkości opisujących układ podczas i po zderzeniu z prawym brzegiem.}
		\label{nptv_2}
	\end{figure}

	\noindent Gdy fala dociera do prawego brzegu zaczynają się tam gromadzić cząsteczki. 
	Tak że widzimy pewne maksimum lokalne, będące nową falą poruszającą się w lewą stronę, gdyż są one \textit{wypychane} przez atomy poruszające się w prawo.
	Takie zachowanie powoduje powstanie dużego ciśnienia w tym obszarze, podobnego do ciśnienia występującego na początku symulacji w lewej części zbiornika.
	\\
	\\
	Ciekawe zachowanie wykazuje temperatura gazu, gdyż jest ona większa niż początkowa temperatura lewego zbiornika. 
	Prędkość natomiast po zderzeniu fali z barierą ponownie się zeruje po prawej stronie od czoła fali, a jest duża na czole fali i po jej lewej stronie (czyli prawdopodobnie cząsteczki wciąż poruszające się w prawo). 
	Jest to ciekawe i raczej nieoczywiste zachowanie. 

	\begin{figure}[H]
		\centering
		\includegraphics[width=\textwidth]{../plots/nptv_1801.pdf}
		\caption{Rozkłady wielkości opisujących układ po zderzeniu z prawym brzegiem.}
		\label{nptv_3}
	\end{figure}
	
	\noindent W kolejnych iteracjach coraz większa ilość cząstek dociera do prawej strony układu, co się bezpośrednio przekłada na większą gęstość i ciśnienie cząsteczek. 
	Na prawym brzegu duża gęstość i ciśnienie się utrzymują, aż ostatecznie zaczynają powoli spadać i się stabilizować
	Ciekawe zachowanie można zauważyć w temperaturze, która jest jeszcze większa na prawym brzegu. 
	Natomiast na lewym brzegu jest dużo niższa.
	Wciąż obserwujemy potężną falę termiczną, tym razem poruszającą się w lewą stronę. 
	\\
	\\
	Prędkość cząstek wykazuje dużą zależność od położenia fali, gdyż w późniejszych etapach jest ona bliska zeru na czole fali.
	Mniej więcej od momentu, w którym czoło fali przeszło środek układu symulacyjnego prędkość zaczyna wzrastać liniowo w układzie, tak że cały czas pozostaje najmniejsza przy prawym brzegu.
	Jednak część cząsteczek z prawego brzegu porusza się teraz prawdopodobnie w lewą stronę. 
	
	\section*{Podsumowanie}
	
	Wykonana symulacja pozwoliła sprawdzić przebieg fali termobarycznej. 
	Zarówno poruszająca się fala ciśnienia jak i temperatury przeniosły się z jednej strony układu na drugą. 
	Wykazały one jednak inne zachowanie w po zderzeniu się z brzegiem, gdzie temperatura znacząco wzrosła ponad jej wartość początkową w lewym brzegu.
	Taka symulacja mimo dużego przybliżenia prawdopodobnie oddaje elementy fali termobarycznej i pokazuje jej duże możliwości.
	Zastosowania takiej fali znajdują się najprawdopodobniej w przemyśle militarnym, gdyż jest to sposób aby zgromadzoną temperaturę i ciśnienie łatwo przetransportować.
	Jest to po prostu wybuch o dużej mocy. 
	
\end{document}