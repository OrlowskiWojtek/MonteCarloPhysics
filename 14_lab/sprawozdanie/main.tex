\documentclass[12pt, a4paper]{article}

\usepackage[utf8]{inputenc}
\usepackage[MeX]{polski}
\usepackage[table]{xcolor}
\usepackage{enumerate}
\usepackage{graphicx}
\usepackage{float}
\usepackage{subcaption}
\usepackage{array}
\usepackage{amsmath}
\usepackage{amssymb}
\usepackage{mathtools}
\usepackage{tabularx}
\usepackage{booktabs}
\usepackage{multirow}
\usepackage{multicol}
\usepackage{hyperref}
\usepackage{physics}
\usepackage{geometry}
\usepackage{bbold}


\newgeometry{tmargin=2cm, bmargin=1.5cm, lmargin=2cm, rmargin=2cm}

\usepackage{sectsty}
\definecolor{ColorOne}{RGB}{0, 0, 0}
\sectionfont{\color{ColorOne}}
\subsectionfont{\color{ColorOne}}
\subsubsectionfont{\color{ColorOne}}
\hypersetup{
  colorlinks=true,
  linkcolor=ColorOne,
  filecolor=ColorOne,
  urlcolor=ColorOne,
  citecolor=ColorOne,
}

\author{Wojciech Orłowski}
\date{\today}
\title{\textsc{Wariacyjne Kwantowe Monte Carlo dla atomu wodoru} - sprawozdanie}

\begin{document}
    \maketitle

    \section*{Wstęp}

    Istnieje wiele metod obliczeniowych pozwalających obliczyć mechanikę nanoświata, w którym rządzi mechanika kwantowa.
    Jedną z metod pozwalających oszacować wartość energii i postać funkcji falowej jest wariacyjna kwantowa metoda Monte Carlo.
    Opiera się ona na metodzie wariacyjnej, w której możemy ograniczyć oddolnie wartość energii dla rozwiązania równania Schrodingera \eqref{schrodinger}..
    \begin{equation}
        \hat{H} \ket{\psi_n} = E_n \ket{\psi_n}.
        \label{schrodinger}
    \end{equation}
    Wprowadźmy pewien stan próbny $\ket{\tilde{\psi}}$.
    Energia, będąca właściwie funkcjonałem od stanu, dla stanu próbnego będzie zapisana poprzez:
    \begin{equation}
        \varepsilon[\tilde{\psi}] = \frac{\bra{\tilde{\psi}}\hat{H}\ket{\tilde{\psi}}}{\bra{\tilde{\psi}}\ket{\tilde{\psi}}}
    \end{equation}
    Wartość oczekiwana hamiltonianu w stanie próbnym $\ket{\tilde{\psi}}$ wynosi:
    \begin{equation}
        \bra{\tilde{\psi}} \hat{H} \ket{\tilde{\psi}} = \bra{\tilde{\psi}} \hat{H} \mathbb{1} \ket{\tilde{\psi}},
        \label{exp_value}
    \end{equation}
    korzystając z zależności zagadnienia własnego \eqref{schrodinger}, w którym wektory własne tworzą bazę zupełną można zapisać
    \begin{equation}
        \bra{\tilde{\psi}} \hat{H} \mathbb{1} \ket{\tilde{\psi}} = \bra{\tilde{\psi}} \hat{H} \sum_{n} \ket{\psi_n}\bra{\psi_n}  \ket{\tilde{\psi}}
        = \sum_n E_n \abs{\bra{\psi_n}\ket{\tilde{\psi}}}^2.
    \end{equation}
    Energia stanu podstawowego jest najniższą z energii $E_n$, dlatego prawdziwa będzie relacja
    \begin{equation}
        \sum_n E_n \abs{\bra{\psi_n}\ket{\tilde{\psi}}}^2 \geq E_0 \sum_n \abs{\bra{\psi_n}\ket{\tilde{\psi}}}^2 = E_0 \sum_n \bra{\tilde{\psi}}\ket{\psi_n}\bra{\psi_n}\ket{\tilde{\psi}} = E_0 \bra{\tilde{\psi}}\ket{\tilde{\psi}},
    \end{equation}
    przenosząc otrzymaną relację, do relacji wejściowej z \eqref{exp_value} można dojść do
    \begin{equation}
        E_0 \leq \frac{\bra{\tilde{\psi}}\hat{H}\ket{\tilde{\psi}}}{\bra{\tilde{\psi}}\ket{\tilde{\psi}}} = \varepsilon[\tilde{\psi}].
    \end{equation}
    Jest to bardzo ważny wynik, gdyż oznacza on że dla funkcji stanu podstawowego istnieje minimum lokalne energii w dziedzinie dostępnych stanów.
    Wybierając za funkcję próbną dowolną sparametryzowaną funkcję możemy znaleźć taki zestaw parametrów, dla którego $\varepsilon[\tilde{\psi}]$ jest minimum.
    Jest duże prawdopodobieństwo, że stan odpowiadający tej funkcji jest stanem podstawowym.
    W metodach wykorzystujących metodę wariacyjną funkcjonał $\varepsilon[\tilde{\psi}]$ staje się funkcją parametrów wariacyjnych, a całe zagadnienie polega na znajdywaniu minimum tej funkcji.
    \\
    \\
    Metodę Monte Carlo wykorzystamy aby oszacować wartość energii dla konkretnych parametrów wariacyjnych, czyli obliczyć wartość wyrażenia
    \begin{equation}
        \varepsilon[\tilde{\psi}] := \varepsilon(c_1,c_2,\dots,c_l) =  \frac{\bra{\tilde{\psi}}\hat{H}\ket{\tilde{\psi}}}{\bra{\tilde{\psi}}\ket{\tilde{\psi}}} 
        = \frac{\int \dd x \, \tilde{\psi}^*(x)\hat{H}\tilde{\psi}(x)  }{\int \dd x \, \tilde{\psi}^*(x) \tilde{\psi}(x)}.
    \end{equation}
    Energię w ten sposób możemy obliczyć energię dla dowolnego stanu wyrażonego funkcją falową $\phi(x)$.
    \begin{equation}
        E = \frac{\int \dd x \, \phi^*(x) \hat{H} \phi(x)}{\int \dd x\, \phi^*(x)\phi(x)} = \frac{\int \dd x \, \abs{\phi(x)}^2 \frac{\hat{H} \phi(x)}{\phi(x)}}{\int \dd x \, \abs{\phi(x)}^2},
        \label{energy}
    \end{equation} 
    takie wyrażenie też możemy zapisać korzystając z pojęć funkcji gęstości prawdopodobieństwa oraz energii lokalnej
    \begin{gather}
        P(x) = \frac{\abs{\phi(x)}^2}{\int \dd x \, \abs{\phi(x)}^2}, \\
        E_{\text{lok}}(x) = \frac{\hat{H}\phi(x)}{\phi(x)},  
    \end{gather}
    co podstawiając do wyrażenia na energie \eqref{energy} daje
    \begin{equation}
        E = \int \dd x \, P(x) E_{\text{lok}}(x).
    \end{equation} 
    Korzystanie z tych pojęć nie jest jedynie trikiem matematycznym pozwalającym intepretować dane zagadnienie jako stochastyczne, lecz jest ono uzasadnione także probabilistyczną interpretacją funkcji falowej.
    Energia układu jest wartością oczekiwaną energii lokalnej.
    Jedną z ważnych zalet wynikających z tej interpretacji funkcji falowej jest możliwość obliczenia tej całki metodą Monte Carlo korzystając z przypadkowego błądzenia lub odpowiednich łancuchów.
    Całe zagadnienie polega na tym aby wylosować daną ilość punktów o rozkładzie zadanym poprzez funkcję gęstości prawdopodobieństwa.
    W ćwiczeniu skorzystano z algorytmu metropolisa w którym nowe punkty losowane są na podstawie pewnego prawdopodobieństwa akceptacji.
    Prawdopodbieństwo akceptacji nowego położenia jest wyrażone poprzez
    \begin{equation}
        p_{\text{acc}} =\min\left[\frac{P(x_{\text{new}})}{P(x)},1 \right] = \min\left[\abs{\frac{\phi(x_{\text{new}})}{\phi(x)}}^2,1 \right].
    \end{equation}
    Energia stanu jest szacowana na podstawie uśrednienie wartości energii lokalnej po wylosowanych położeniach podczas błądzenia losowego
    \begin{equation}
        E = \frac{1}{N} \sum_{i=1}^N E_{\text{lok}}(x_i),
    \end{equation}
    a wariancję jako różnicę pierwszego i drugiego momentu. Dla rozwiązania dokładnego, w którym energia jest energią szukanego $n$-tego stanu otrzymamy
    \begin{equation}
        \text{var}(\hat{H}) = \frac{\bra{\phi} (\hat{H} - E)^2 \ket{\phi}}{\bra{\phi}\ket{\phi}} =\int \dd x \, P(x) \abs{E_{\text{lok}}(x) - E} = \frac{1}{N} \sum_{i=1}^N \left( E_{\text{lok}}(x_i) - E \right) = 0,
    \end{equation}
    zatem dla szukanego stanu wartość wariancji się zeruje.
    \\
    \\
    W zadanym programie wykonujemy obliczenia dotyczące atomu wodoru.
    Rozważania dotyczące mechaniki atomu wodoru rozpoczynamy od wprowadzenia odpowiedniego układu współrzędnych sferycznych, gdzie możemy dokonać separacji zmiennych.
    Część radialna hamiltonianu ma postać
    \begin{equation}
        \hat{H} = -\frac{1}{2} \left[ \frac{1}{r^2} \pdv{r} r^2 \pdv{r} - \frac{l(l+1)}{r^2} \right] - \frac{1}{r}.
    \end{equation}
    Wprowadzenie nowego układu współrzędnych wymagało też dokonania zmiany w całce, za pomocą której liczymy energię - funkcja podcałkowa jest przemnożona przez odpowiedni Jakobian przejścia.
    Nowa funkcja gęstości prawdopodobieństwa, której zmiana jest związana z Jakobianem jest zdefiniowana jako
    \begin{equation}
        P(r) = \frac{r^2\abs{\Psi_T(r)}^2}{\int \dd r \, \abs{\Psi_T(r)}^2},
    \end{equation}
    a wartość energii lokalnej pozostaje bez zmian.
    Należy natomiast pamiętać, że zmiana wartości funkcji gęstości prawdopodobieństwa powoduje zmianę prawdopodobieństwa akcpetacji nowego położenia.
    Znamy dokładne rozwiązania dla atomu wodoru, dlatego możemy zapostulować taką funkcję próbną, która dla odpowiednich parametrów da wartość dokładną dla obu szukanych przypadków (stanu podstawowego i pierwszego stanu wzbudzonego).
    Zatem deklarujemy funkcję próbną zależną od dwóch parametrów wariacyjnych
    \begin{equation}
        \Psi_T(r) = (1 + cr)\exp(-a r).
    \end{equation}
    Stan podstawowy i pierwszy wzbudzony są wyrażone poprzez 
    \begin{equation}
        \begin{array}{c ccc}
           \text{podstawowy} & a = 1; & c = 0; & E = -\frac{1}{2} \\
           \text{wzbudzony} & a = \frac{1}{2} & c = -\frac{1}{2} & E = -\frac{1}{8} 
        \end{array}
    \end{equation}
    Obliczona energia lokalna dla wybranej funkcji próbnej ma postać 
    \begin{equation}
        E_{\text{lok}} = \frac{-a^2cr^2 + (-a^2 + 4ac -2c)r +2a -2c -2}{2cr^2 + 2r}.
    \end{equation}
    Zatem znając wartość energii dla danych parametrów $a,c$ jesteśmy w stanie ją zminimalizować.
    W ćwiczeniu nie użyto żadnego algorytmu minimalizującego, lecz stworzono całą mapę energii w zależności parametrów $a$ i $c$.

    \newpage

    \section*{Wyniki}

    W zadaniu przyjęto, że maksymalna liczba kroków losowych wynosi $N = 10^6$.
    Sprawdzono wartości energii oraz jej wariancji dla $a \in [0.3,1.2]$ oraz $c \in [-0.7,0.3]$. 
    Parametry wariacyjne zmieniane były o kroki $\Delta_a = \Delta_c = 0.02$, tak aby z pewnością trafić na parametry dokładne. 
    \\
    \\
    Obliczenia wykonano za pomocą języka programowania \texttt{Julia}. 
    Obliczone wartości energii oraz jej wariancji zostały przedstawione na rys. \ref{maps}.
    \begin{figure}[H]
        \centering
        \includegraphics[width = \textwidth]{../plots/maps.pdf}
        \caption{Obliczone mapy energii, wariancji i logarytmu z wariancji dla atomu wodoru.}
        \label{maps}
    \end{figure}

    \noindent Obliczone energie były do siebie bardzo podobne i ciężko byłoby znaleźć minima lokalne wizualnie.
    Dlatego musimy skorzystać z warunku zerowania się wariancji.
    Jednak duża część mapy wariancji, tak jak na widocznej mapie, jest bliska wartości zerowej, i ciężko odnaleźć gdzie dokładnie znajduje się zero.
    Aby uwidocznić najmniejsze znaleziony wartości skonstruowano trzecią mapę, przedstawiającą wartości logarytmu z wariancji.
    Logarytm z zera jednak wynosi minus nieskończoność
    \[ \lim_{x \to 0} \log(x) = -\infty, \]
    dlatego w celu uniknięcia osobliwości do wartości wariancji sztucznie dodano $10^{-20}$. 
    W ten sposób otrzymano graficznie dwa wyraźne miejsca, w których wariancja się zeruje (ma wartość bardzo bliską zera).
    Graficznie też można odczytać wartości parametrów wariacyjnych, dla których znaleziono stany własne hamiltionianu.
    Są to wartości dokładne. 
    Z wartości energii natomiast dla tych punktów możemy stwierdzić, który z nich jest stanem podstawowym, a który wzbudzonym.
    Tak widzimy, że mniejsze wartości energii znajdują się w prawym górnym rogu mapy, dlatego punkt (a = 1, c = 0) jest uznawany za stan podstawowy.
    W lewym dolnym rogu wartości energii są wyższe, co można rozpoznać po samym kolorze mapy.
    Dla wartości parametrów wariacyjnych (a = 0.5, c = -0.5) znaleziono pierwszy stan wzbudzony.
    \\
    \\
    Zgodnie z wcześniej wprowadzoną teorią wiemy, że funkcja gęstości prawdopodobieństwa odpowiadająca danej energii lokalnej wynosi
    \[ P(r) = r^2 \abs{\Psi_T(r)}^2, \]
    z dokładnością do unormowania.
    Całkując schematem metropolisa powinniśmy wziąć pod uwagę ilość punktów zgodną z tym rozkładem.
    Dodatkowo dla dokładnej funkcji falowej rozkład wylosowanych punktów powinien odpowiadać wartości rozkładu prawdopodobieństwa dla stanu dokładnego.
    Rozkład wylosowanych punktów wraz z naniesionym teoretycznym rozkładem prawdopodobieńśtwa dla funkcji stanu podstawowego został przedstwiony na rys. \ref{hist}

    \begin{figure}[H]
        \centering
        \includegraphics[width = 0.7\textwidth]{../plots/hist.pdf}
        \caption{Unormowany histogram wylosowanych punktów wraz z teoretyczną funkcją gęstości prawdopodobieństwa dla stanu podstawowego.}
        \label{hist}
    \end{figure}

    \noindent Dla $n=10^6$ widzimy, że obie krzywe są bardzo podobne do siebie kształtem.
    Jedyne większe odchylenia można zaobserwować dla maksimum tej krzywej, czyli w miejscu gdzie punkty powinny być najczęściej losowane.
    Sprawdzenie dokładności dopasowania może stanowić metodę walidacji wyników.
    Odchylenia nie świadczą jednak o błędzie w obliczeniach, tak jak w tym przypadku uzyskano wartość energii równą dokładnej ($E = - \frac{1}{2}$).    

    \section*{Podsumowanie}

    Prosty algorytm całkowania pozwolił na zapoznanie się z ideą kwantowego wariacyjnego Monte Carlo.
    Metoda ta ma wiele zalet - jest szybka, dokładna, opiera się na dobrze opracowanej i prostej metodzie wariacyjnej oraz może mieć wiele modyfikacji.
    Podstawową modyfikacją wydaje się implementacja algorytmu minimalizującego, tak aby nie trzeba było obliczać wartości energii tak dużą ilość razy.
    Metodę tą można też rozszerzyć do układów wielociałowych bazując na metodzie Hartree-Focka i wyznaczników Slatera.
    \\
    \\
    Dodatkowym bonusem korzystania z tej metody w aspekcie mechaniki kwantowej jest łatwa możliwość oszacowania wariancji, a co za tym idzie błędu.
    Jak wiadomo, funkcję falowe muszą być klasy $L_2$ (całkowalne w kwadracie).
    Aby obliczyć wartość wariancji należy obliczyć wartość oczekiwaną kwadratu zmiennej losowej - zatem też należy obliczyć całkę kwadratu funkcji.
    Jest to możliwe dla funkcji klasy $L_2$. 



\end{document}
